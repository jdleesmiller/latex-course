\documentclass[12pt]{article}
\usepackage[english,francais]{babel}
\usepackage[T1]{fontenc}
\usepackage{url,lmodern}

\title{Dix secrets pour réussir un bon exposé scientifique}
\author{Vous}

\begin{document}
\maketitle

\section{Introduction}

Le texte de cet exercice est une version significativement abrégée et légèrement modifié
de l'excellent article ainsi intitulé par Mark Schoeberl et Brian Toon :
\url{http://www.cgd.ucar.edu/cms/agu/scientific_talk.html}

\section{Les secrets}

J'ai compilé cette liste personnelle de "secrets" en écoutant des orateurs efficaces et inefficaces. Je ne prétends pas que cette liste est exhaustive --- je suis sûr qu'il y a des choses que j'ai oubliées. Mais ma liste comporte probablement 90\% de ce que vous devez savoir et faire.

\begin{enumerate}

\item Préparez votre matériel attentivement et avec logique. Racontez une histoire.

\item Faites des répétitions. Il n'y pas d'excuse pour un manque de préparation.

\item N'y insérez pas trop de matière. Les bons orateurs choisissent un ou deux points importants et restent là-dessus.

\item Évitez les équations. On dit que pour chaque équation dans votre présentation, le nombre d'auditeurs qui vont comprendre sera divisé par deux. Autrement dit, si $q$ est le nombre d'équations dans votre présentation et $n$ le nombre de personnes qui la comprennent, alors 
\begin{equation}
n = \gamma \left( \frac{1}{2} \right)^q
\end{equation}
où $\gamma$ est une constante de proportionnalité.

\item Ne gardez que peu de points de conclusion. Les gens ne peuvent se souvenir que de très peu de choses surtout si elles suivent plusieurs exposés dans des grandes réunions.

\item Adressez-vous à votre auditoire et pas à l'écran. Un des problèmes les plus fréquents est que l'orateur s'adresse à l'écran de projection.

\item Évitez les sons distrayants. Évitez les \og Hummm\fg{} ou \og Ahhh\fg{} entre deux phrases.

\item Soignez vos graphiques. Voici une liste de tuyaux pour les bons graphiques :

\begin{itemize}
\item Utilisez des gros caractères.

\item Gardez les graphiques simples. Ne montrer pas les graphiques dont vous n'avez pas besoin.

\item Utilisez la couleur.

\end{itemize}

\item Soyez charmant quand vous répondez aux questions.

\item Utilisez l'humour si possible. Je suis toujours à quel point une blague même lamentable peut déclencher un bon rire lors d'un exposé scientifique.

\end{enumerate}

\end{document}
