\documentclass{beamer}

%
% Common preamble for all three parts.
%

\usepackage[english]{babel}
\usepackage{amsmath}
\usepackage{color}
\usepackage{minted}
\usepackage{hyperref}
\usepackage{multicol}
\usepackage{tabularx}
\usepackage{tikz}

% only inline todonotes work
\usepackage{xkeyval}
\usepackage[textsize=small]{todonotes}
\presetkeys{todonotes}{inline}{}

\usetikzlibrary{shapes,arrows,positioning,shadows}

% no nav buttons
\usenavigationsymbolstemplate{}

\newcommand{\bftt}[1]{\textbf{\texttt{#1}}}
\newcommand{\comment}[1]{{\color[HTML]{008080}\textit{\textbf{\texttt{#1}}}}}
\newcommand{\cmd}[1]{{\color[HTML]{008000}\bftt{#1}}}
\newcommand{\bs}{\char`\\}
\newcommand{\cmdbs}[1]{\cmd{\bs#1}}
\newcommand{\lcb}{\char '173}
\newcommand{\rcb}{\char '175}
\newcommand{\cmdbegin}[1]{\cmdbs{begin\lcb}\bftt{#1}\cmd{\rcb}}
\newcommand{\cmdend}[1]{\cmdbs{end\lcb}\bftt{#1}\cmd{\rcb}}

\newcommand{\wllogo}{\textbf{write\textrm{\LaTeX}}}

% this is where the example source files are loaded from
% do not include a trailing slash
\newcommand{\fileuri}{https://raw.github.com/jdleesmiller/latex-course/master/en}

\newcommand{\wlserver}{https://www.writelatex.com}
\newcommand{\wlnewdoc}[1]{\wlserver/docs?snip\_uri=\fileuri/#1\&splash=none}

\def\tikzname{Ti\emph{k}Z}

% from http://tex.stackexchange.com/questions/5226/keyboard-font-for-latex
\newcommand*\keystroke[1]{%
  \tikz[baseline=(key.base)]
    \node[%
      draw,
      fill=white,
      drop shadow={shadow xshift=0.25ex,shadow yshift=-0.25ex,fill=black,opacity=0.75},
      rectangle,
      rounded corners=2pt,
      inner sep=1pt,
      line width=0.5pt,
      font=\scriptsize\sffamily
    ](key) {#1\strut}
  ;
}
\newcommand{\keystrokebftt}[1]{\keystroke{\bftt{#1}}}

% stolen from minted.dtx
\newenvironment{exampletwoup}
  {\VerbatimEnvironment
   \begin{VerbatimOut}{example.out}}
  {\end{VerbatimOut}
   \setlength{\parindent}{0pt}
   \fbox{\begin{tabular}{l|l}
   \begin{minipage}{0.55\linewidth}
     \inputminted[fontsize=\small,resetmargins]{latex}{example.out}
   \end{minipage} &
   \begin{minipage}{0.35\linewidth}
     \input{example.out}
   \end{minipage}
   \end{tabular}}}

\newenvironment{exampletwouptiny}
  {\VerbatimEnvironment
   \begin{VerbatimOut}{example.out}}
  {\end{VerbatimOut}
   \setlength{\parindent}{0pt}
   \fbox{\begin{tabular}{l|l}
   \begin{minipage}{0.55\linewidth}
     \inputminted[fontsize=\scriptsize,resetmargins]{latex}{example.out}
   \end{minipage} &
   \begin{minipage}{0.35\linewidth}
     \setlength{\parskip}{6pt plus 1pt minus 1pt}%
     \raggedright\scriptsize\input{example.out}
   \end{minipage}
   \end{tabular}}}

\newenvironment{exampletwouptinynoframe}
  {\VerbatimEnvironment
   \begin{VerbatimOut}{example.out}}
  {\end{VerbatimOut}
   \setlength{\parindent}{0pt}
   \begin{tabular}{l|l}
   \begin{minipage}{0.55\linewidth}
     \inputminted[fontsize=\scriptsize,resetmargins]{latex}{example.out}
   \end{minipage} &
   \begin{minipage}{0.35\linewidth}
     \setlength{\parskip}{6pt plus 1pt minus 1pt}%
     \raggedright\scriptsize\input{example.out}
   \end{minipage}
   \end{tabular}}

\title{An Interactive Introduction to \LaTeX}
\author{Dr John D. Lees-Miller}
\titlegraphic{%
\includegraphics[page=4]{wllogo-series}\\[1em]
\includegraphics[height=24pt]{UoB-logo}
\qquad
\includegraphics[height=24pt]{setsquared_supported}
}



\subtitle{Première parte : les bases}

\begin{document}

%%%%%%%%%%%%%%%%%%%%%%%%%%%%%%%%%%%%%%%%%%%%%%%%%%%%%%%%%%%%%%%%%%%%%%%%%%%%%%%
%%%%%%%%%%%%%%%%%%%%%%%%%%%%%%%%%%%%%%%%%%%%%%%%%%%%%%%%%%%%%%%%%%%%%%%%%%%%%%%
%%%%%%%%%%%%%%%%%%%%%%%%%%%%%%%%%%%%%%%%%%%%%%%%%%%%%%%%%%%%%%%%%%%%%%%%%%%%%%%
\begin{frame}
\titlepage
\end{frame}

%%%%%%%%%%%%%%%%%%%%%%%%%%%%%%%%%%%%%%%%%%%%%%%%%%%%%%%%%%%%%%%%%%%%%%%%%%%%%%%
%%%%%%%%%%%%%%%%%%%%%%%%%%%%%%%%%%%%%%%%%%%%%%%%%%%%%%%%%%%%%%%%%%%%%%%%%%%%%%%
%%%%%%%%%%%%%%%%%%%%%%%%%%%%%%%%%%%%%%%%%%%%%%%%%%%%%%%%%%%%%%%%%%%%%%%%%%%%%%%
\begin{frame}{Pourquoi \LaTeX{} ?}
\begin{itemize}
\item Il produit des beaux documents
\begin{itemize}
\item Et plus particulièrement quand ils contiennent des mathématiques
\end{itemize}
%
\item Il a été créé par des scientifiques pour des scientifiques
\begin{itemize}
\item Une communauté large et active
\end{itemize}
%
\item Il est puissant --- vous pouvez l'étendre
\begin{itemize}
\item Il comporte des packages pour les publis, les présentations, les tableurs...
\end{itemize}
\end{itemize}
\end{frame}

%%%%%%%%%%%%%%%%%%%%%%%%%%%%%%%%%%%%%%%%%%%%%%%%%%%%%%%%%%%%%%%%%%%%%%%%%%%%%%%
%%%%%%%%%%%%%%%%%%%%%%%%%%%%%%%%%%%%%%%%%%%%%%%%%%%%%%%%%%%%%%%%%%%%%%%%%%%%%%%
%%%%%%%%%%%%%%%%%%%%%%%%%%%%%%%%%%%%%%%%%%%%%%%%%%%%%%%%%%%%%%%%%%%%%%%%%%%%%%%
\begin{frame}[fragile]{Comment fonctionne-t-il ?}
\begin{itemize}
\item Vous écrivez votre document en \texttt{texte brut} parsemé de \cmd{commandes} qui
décrivent sa structure et son contenu.
\item Le programme \texttt{latex} traite votre texte et vos commandes, pour produire un document magnifiquement présenté.
\end{itemize}
\vskip 2ex
\begin{center}
\begin{minted}[frame=single]{latex}
La plume est plus \emph{forte} que l'épée.
\end{minted}
\vskip 2ex
\tikz\node[single arrow,fill=gray,font=\ttfamily\bfseries,%
  rotate=270,xshift=-1em]{latex};
\vskip 2ex
\fbox{La plume est plus \emph{forte} que l'épée.}
\end{center}
\end{frame}

%%%%%%%%%%%%%%%%%%%%%%%%%%%%%%%%%%%%%%%%%%%%%%%%%%%%%%%%%%%%%%%%%%%%%%%%%%%%%%%
%%%%%%%%%%%%%%%%%%%%%%%%%%%%%%%%%%%%%%%%%%%%%%%%%%%%%%%%%%%%%%%%%%%%%%%%%%%%%%%
%%%%%%%%%%%%%%%%%%%%%%%%%%%%%%%%%%%%%%%%%%%%%%%%%%%%%%%%%%%%%%%%%%%%%%%%%%%%%%%
\begin{frame}[fragile]{Plus d'exemples de commandes et de leurs sorties...}
\begin{exampletwoup}
\begin{itemize}
\item Thé
\item Lait
\item Biscuits
\end{itemize}
\end{exampletwoup}
\vskip 2ex
\begin{exampletwoup}
\begin{figure}
\includegraphics{gerbil}
\end{figure}
\end{exampletwoup}
\vskip 2ex
\begin{exampletwoup}
\begin{equation}
\alpha + \beta + 1
\end{equation}
\end{exampletwoup}

\tiny{Droit d'auteur de l'image : \href{https://pixabay.com/en/animal-apple-attractive-beautiful-1239390/}{CC0}}
\end{frame}

%%%%%%%%%%%%%%%%%%%%%%%%%%%%%%%%%%%%%%%%%%%%%%%%%%%%%%%%%%%%%%%%%%%%%%%%%%%%%%%
%%%%%%%%%%%%%%%%%%%%%%%%%%%%%%%%%%%%%%%%%%%%%%%%%%%%%%%%%%%%%%%%%%%%%%%%%%%%%%%
%%%%%%%%%%%%%%%%%%%%%%%%%%%%%%%%%%%%%%%%%%%%%%%%%%%%%%%%%%%%%%%%%%%%%%%%%%%%%%%
\begin{frame}[fragile]{Changement d'attitude}

\begin{itemize}
\item Utilisez des commandes qui décrivent \og{}ce que c'est\fg{} et non pas \og{}à quoi il doit ressembler\fg{}.
\item Focalisez-vous sur votre contenu.
\item Laissez \LaTeX{} faire son travail.
\end{itemize}
\end{frame}

%%%%%%%%%%%%%%%%%%%%%%%%%%%%%%%%%%%%%%%%%%%%%%%%%%%%%%%%%%%%%%%%%%%%%%%%%%%%%%%
%%%%%%%%%%%%%%%%%%%%%%%%%%%%%%%%%%%%%%%%%%%%%%%%%%%%%%%%%%%%%%%%%%%%%%%%%%%%%%%
%%%%%%%%%%%%%%%%%%%%%%%%%%%%%%%%%%%%%%%%%%%%%%%%%%%%%%%%%%%%%%%%%%%%%%%%%%%%%%%
\section{Les bases}

%%%%%%%%%%%%%%%%%%%%%%%%%%%%%%%%%%%%%%%%%%%%%%%%%%%%%%%%%%%%%%%%%%%%%%%%%%%%%%%
%%%%%%%%%%%%%%%%%%%%%%%%%%%%%%%%%%%%%%%%%%%%%%%%%%%%%%%%%%%%%%%%%%%%%%%%%%%%%%%
%%%%%%%%%%%%%%%%%%%%%%%%%%%%%%%%%%%%%%%%%%%%%%%%%%%%%%%%%%%%%%%%%%%%%%%%%%%%%%%
\subsection{Pour commencer}
\begin{frame}[fragile]{\insertsubsection}
\begin{itemize}
\item Un document \LaTeX{} minimal :
\inputminted[frame=single]{latex}{basics.tex}
\item Les commandes commencent par un \emph{antislash} \keystrokebftt{\bs}.
\item Chaque document commence par une commande \cmdbs{documentclass}.
\item L'\emph{argument} entre accolades \keystrokebftt{\{} \keystrokebftt{\}} informe \LaTeX{} sur le type de document vous êtes en train de créer : un \bftt{article}.
\item Un signe pourcent \keystrokebftt{\%} commence un \emph{commentaire} --- \LaTeX{}
va ignorer le reste de la ligne.
\end{itemize}
\end{frame}

%%%%%%%%%%%%%%%%%%%%%%%%%%%%%%%%%%%%%%%%%%%%%%%%%%%%%%%%%%%%%%%%%%%%%%%%%%%%%%%
%%%%%%%%%%%%%%%%%%%%%%%%%%%%%%%%%%%%%%%%%%%%%%%%%%%%%%%%%%%%%%%%%%%%%%%%%%%%%%%
%%%%%%%%%%%%%%%%%%%%%%%%%%%%%%%%%%%%%%%%%%%%%%%%%%%%%%%%%%%%%%%%%%%%%%%%%%%%%%%
\begin{frame}[fragile]{\insertsubsection{} with \wllogo}
\begin{itemize}
\item Overleaf est un site Web pour écrire des documents en \LaTeX.
\item Il \og{}compile\fg{} votre code \LaTeX{} automatiquement et vous montre les résultats.
\vskip 2em
\begin{center}
\fbox{\href{\wlnewdoc{basics.tex}}{%
Cliquer ici pour ouvrir l'exemple de document dans \wllogo{}}}
\\[1ex]\scriptsize{}
Pour obtenir les meilleurs résultats possibles, utilisez \href{http://www.google.com/chrome}{Google Chrome} ou un \href{http://www.mozilla.org/en-US/firefox/new/}{FireFox} récent.
\end{center}
\vskip 2ex
\item En parcourant les transparents suivants, essayez les exemples en les tapant dans le document d'exemple sur Overleaf.
\item \textbf{Non, vraiment, vous devriez les essayer pendant que nous avançons !}
\end{itemize}
\end{frame}

%%%%%%%%%%%%%%%%%%%%%%%%%%%%%%%%%%%%%%%%%%%%%%%%%%%%%%%%%%%%%%%%%%%%%%%%%%%%%%%
%%%%%%%%%%%%%%%%%%%%%%%%%%%%%%%%%%%%%%%%%%%%%%%%%%%%%%%%%%%%%%%%%%%%%%%%%%%%%%%
%%%%%%%%%%%%%%%%%%%%%%%%%%%%%%%%%%%%%%%%%%%%%%%%%%%%%%%%%%%%%%%%%%%%%%%%%%%%%%%
\subsection{Composition de texte}
\begin{frame}[fragile]{\insertsubsection{}}
\small
\begin{itemize}
\item Tapez votre texte entre \cmdbegin{document} et \cmdend{document}.
\item Le plus souvent vous pouvez taper votre texte normalement.
\begin{exampletwouptiny}
Les mots sont séparés par
un ou plusieurs espaces.

Les paragraphes sont séparés
par une ou plusieurs lignes blanches.
\end{exampletwouptiny}
\item Quelque soit le nombre de blancs consécutifs dans votre code source, dans la sortie vous n'aurez qu'un seul blanc.
\begin{exampletwouptiny}
La   plume       est plus forte
que l'épée.
\end{exampletwouptiny}
\end{itemize}
\end{frame}

%%%%%%%%%%%%%%%%%%%%%%%%%%%%%%%%%%%%%%%%%%%%%%%%%%%%%%%%%%%%%%%%%%%%%%%%%%%%%%%
%%%%%%%%%%%%%%%%%%%%%%%%%%%%%%%%%%%%%%%%%%%%%%%%%%%%%%%%%%%%%%%%%%%%%%%%%%%%%%%
%%%%%%%%%%%%%%%%%%%%%%%%%%%%%%%%%%%%%%%%%%%%%%%%%%%%%%%%%%%%%%%%%%%%%%%%%%%%%%%
\begin{frame}[fragile]{\insertsubsection{} : points délicats}
\small
\begin{itemize}
\item Les guillemets américains sont particuliers : 
tapez un accent grave \keystroke{\`{}} à gauche et une apostrophe \keystroke{\'{}} à droite du mot.
\begin{exampletwouptiny}
Guillemets américains
simples : `text'.

Guillemets américains
doubles : ``text''.
\end{exampletwouptiny}

\item Quelques caractères fréquents ont un sens particulier sous \LaTeX:\\[1ex]
\begin{tabular}{cl}
\keystrokebftt{\%} & le signe pourcent                 \\
\keystrokebftt{\#} & la dièse                       \\
\keystrokebftt{\&} & l'esperluette                  \\
\keystrokebftt{\$} & le signe de dollar             \\
\end{tabular}
\item En les tapant directement vous aurez une erreur. 

Si vous voulez les \mbox{obtenir} dans la sortie, il faut les \emph{protéger} en les précédant par un antislash.
\begin{exampletwoup}
\$\%\&\#!
\end{exampletwoup}
\end{itemize}
\end{frame}

%%%%%%%%%%%%%%%%%%%%%%%%%%%%%%%%%%%%%%%%%%%%%%%%%%%%%%%%%%%%%%%%%%%%%%%%%%%%%%%
%%%%%%%%%%%%%%%%%%%%%%%%%%%%%%%%%%%%%%%%%%%%%%%%%%%%%%%%%%%%%%%%%%%%%%%%%%%%%%%
%%%%%%%%%%%%%%%%%%%%%%%%%%%%%%%%%%%%%%%%%%%%%%%%%%%%%%%%%%%%%%%%%%%%%%%%%%%%%%%
\begin{frame}[fragile]{Pour écrire en français (rédigé par Yannis Haralambous)}
\small
\begin{itemize}
\item Pour écrire des documents en français, inclure dans le préambule :
\begin{verbatim}
\usepackage[english,french]{babel}
\usepackage[T1]{fontenc}
\end{verbatim}
\item Voici quelques commandes utiles (qui parlent d'elles-mêmes) :
\begin{exampletwouptiny}
Nos chers \og{}guillemets\fg{}.

{\OE}il, c{\oe}ur, b{\oe}uf...

Le n\up{o}~13 et les n\up{os}~5 et 6,
à ne pas confondre avec les 
37,2~\textdegree C, ou l'alcool 
à 80\textdegree.

M\up{me}, M\up{elle}, le 1\up{er}
du mois, était-ce la 1\up{re}, 
la 2\up{e} ou la $n$\up{e}~fois ?
Vive le \textsc{xxi}\up{e}~siècle !
Les nombres s'écrivent 65\,536,
1\,000\,000, et ainsi de suite...
\end{exampletwouptiny}
\item Pour plus d'infos, consultez le \emph{Lexique des règles typographiques en usage à l'Imprimerie nationale}, Paris, Imprimerie nationale, 2006.
\end{itemize}
\end{frame}

%%%%%%%%%%%%%%%%%%%%%%%%%%%%%%%%%%%%%%%%%%%%%%%%%%%%%%%%%%%%%%%%%%%%%%%%%%%%%%%
%%%%%%%%%%%%%%%%%%%%%%%%%%%%%%%%%%%%%%%%%%%%%%%%%%%%%%%%%%%%%%%%%%%%%%%%%%%%%%%
%%%%%%%%%%%%%%%%%%%%%%%%%%%%%%%%%%%%%%%%%%%%%%%%%%%%%%%%%%%%%%%%%%%%%%%%%%%%%%%
\begin{frame}[fragile]{Gestion des erreurs}
\begin{itemize}
\item \LaTeX{} peut ne pas accepter une partie de votre code. Dans ce cas
il va s'arrêter en affichant un message d'erreur, c'est à vous de corriger
cette erreur pour qu'il puisse produire votre document.
\item Par exemple, si vous tapez \cmdbs{meph} au lieu de \cmdbs{emph}, \LaTeX{} s'arrêtera
en affichant un message de \og{}commande non définie\fg{} puisque \og{}meph\fg{} ne fait
pas partie des commandes qu'il connaît.
\end{itemize}
\begin{block}{Conseils concernant les erreurs}
\begin{enumerate}
\item Ne paniquez pas ! Ça arrive à tout le monde de faire des erreurs.
\item Corrigez-les aussitôt qu'elles apparaissent --- si votre saisie provoque une erreur, commencez le débogage à cet endroit.
\item S'il y a plus d'une erreurs, commencez par la première --- la cause peut être située avant-même celle-ci.
\end{enumerate}
\end{block}
\end{frame}

%%%%%%%%%%%%%%%%%%%%%%%%%%%%%%%%%%%%%%%%%%%%%%%%%%%%%%%%%%%%%%%%%%%%%%%%%%%%%%%
%%%%%%%%%%%%%%%%%%%%%%%%%%%%%%%%%%%%%%%%%%%%%%%%%%%%%%%%%%%%%%%%%%%%%%%%%%%%%%%
%%%%%%%%%%%%%%%%%%%%%%%%%%%%%%%%%%%%%%%%%%%%%%%%%%%%%%%%%%%%%%%%%%%%%%%%%%%%%%%
\begin{frame}[fragile]{Exercice de composition 1}

\begin{block}{Composez ceci en \LaTeX\footnote{\url{http://en.wikipedia.org/wiki/Economy_of_the_United_States}}:
}
En mars 2006, le Congrès américain a relevé
ce plafond de 0,79~mille milliards de \$ pour
arriver à 8,97 mille milliards de \$, qui est
environ 68\% du PIB. Le 4 octobre 2008, la loi
de \og{}stabilisation économique urgente\fg{} de
2008 a relevé le plafond actuel de la dette à
11,3 mille milliards de \$.
\end{block}
\vskip 2ex
\begin{center}
\fbox{\href{\wlnewdoc{basics-exercise-1.tex}}{%
Cliquez pour ouvrir cet exercice sous \wllogo{}}}
\end{center}

\begin{itemize}
\item Tuyau : attention aux caractères qui ont une signification spéciale  !
\item Après avoir essayé,
\fbox{\href{\wlnewdoc{basics-exercise-1-solution.tex}}{%
cliquez ici pour voir ma solution}}.
\end{itemize}
\end{frame}

%%%%%%%%%%%%%%%%%%%%%%%%%%%%%%%%%%%%%%%%%%%%%%%%%%%%%%%%%%%%%%%%%%%%%%%%%%%%%%%
%%%%%%%%%%%%%%%%%%%%%%%%%%%%%%%%%%%%%%%%%%%%%%%%%%%%%%%%%%%%%%%%%%%%%%%%%%%%%%%
%%%%%%%%%%%%%%%%%%%%%%%%%%%%%%%%%%%%%%%%%%%%%%%%%%%%%%%%%%%%%%%%%%%%%%%%%%%%%%%
\subsection{Composition de mathématiques}
\begin{frame}[fragile]{\insertsubsection{}: signe de dollar}
\begin{itemize}
\item Pourquoi les signes de dollar \keystrokebftt{\$} sont-ils particuliers ? Nous les utilisons pour marquer les mathématiques dans le texte.\\[1ex]
\begin{exampletwouptiny}
% pas très bien :
Soient a et b des entiers positifs
distincts, et soit c = a - b + 1.

% bien mieux :
Soient $a$ et $b$ des entiers positifs
distincts, et soit $c = a - b + 1$.
\end{exampletwouptiny}
\item Il faut toujours utiliser les signes de dollar par paires --- un signe pour commencer l'expression mathématique, un autre pour la finir.
\item \LaTeX{} gère l'espacement automatiquement ; il ignore vos espaces.
\begin{exampletwouptiny}
Soit $y=mx+b$ la...

Soit $y = m x + b$ la...
\end{exampletwouptiny}
\end{itemize}
\end{frame}

%%%%%%%%%%%%%%%%%%%%%%%%%%%%%%%%%%%%%%%%%%%%%%%%%%%%%%%%%%%%%%%%%%%%%%%%%%%%%%%
%%%%%%%%%%%%%%%%%%%%%%%%%%%%%%%%%%%%%%%%%%%%%%%%%%%%%%%%%%%%%%%%%%%%%%%%%%%%%%%
%%%%%%%%%%%%%%%%%%%%%%%%%%%%%%%%%%%%%%%%%%%%%%%%%%%%%%%%%%%%%%%%%%%%%%%%%%%%%%%
\begin{frame}[fragile]{\insertsubsection{}: notations}
\begin{itemize}
\item Utilisez l'accent \mbox{circonflexe} \keystrokebftt{\^} pour les exposants et le souligné \keystrokebftt{\_} pour les indices.
\begin{exampletwouptiny}
$y = c_2 x^2 + c_1 x + c_0$
\end{exampletwouptiny}
\vskip 2ex

\item Utilisez les accolades \keystrokebftt{\{} \keystrokebftt{\}} pour grouper les exposants et les indices.
\begin{exampletwouptiny}
$F_n = F_n-1 + F_n-2$     % oups !

$F_n = F_{n-1} + F_{n-2}$ % ok !
\end{exampletwouptiny}
\vskip 2ex

\item Il y a des commandes pour les lettres grecques et les symboles.
\begin{exampletwouptiny}
$\mu = A e^{Q/RT}$

$\Omega = \sum_{k=1}^{n} \omega_k$
\end{exampletwouptiny}
\end{itemize}
\end{frame}

%%%%%%%%%%%%%%%%%%%%%%%%%%%%%%%%%%%%%%%%%%%%%%%%%%%%%%%%%%%%%%%%%%%%%%%%%%%%%%%
%%%%%%%%%%%%%%%%%%%%%%%%%%%%%%%%%%%%%%%%%%%%%%%%%%%%%%%%%%%%%%%%%%%%%%%%%%%%%%%
%%%%%%%%%%%%%%%%%%%%%%%%%%%%%%%%%%%%%%%%%%%%%%%%%%%%%%%%%%%%%%%%%%%%%%%%%%%%%%%
\begin{frame}[fragile]{\insertsubsection{}: équations en vedette}
\begin{itemize}
\item Si votre formule est longue et fait peur, \emph{présentez-la} sur une ligne à part en utilisant
\cmdbegin{equation} et \cmdend{equation}.\\[2ex]
\begin{exampletwouptiny}
Les racines d'un polynôme de deuxième
degré sont données par
\begin{equation}
x = \frac{-b \pm \sqrt{b^2 - 4ac}}
         {2a}
\end{equation}
où $a$, $b$ et $c$ sont...
\end{exampletwouptiny}
\vskip 1em
{\scriptsize Attention : même si \LaTeX{} ignore les espaces dans les formules mathématiques, il ne digère pas les lignes vides dans les équations --- n'insérez pas de ligne vide dans vos mathématiques.

}
\end{itemize}
\end{frame}

%%%%%%%%%%%%%%%%%%%%%%%%%%%%%%%%%%%%%%%%%%%%%%%%%%%%%%%%%%%%%%%%%%%%%%%%%%%%%%%
%%%%%%%%%%%%%%%%%%%%%%%%%%%%%%%%%%%%%%%%%%%%%%%%%%%%%%%%%%%%%%%%%%%%%%%%%%%%%%%
%%%%%%%%%%%%%%%%%%%%%%%%%%%%%%%%%%%%%%%%%%%%%%%%%%%%%%%%%%%%%%%%%%%%%%%%%%%%%%%
\begin{frame}[fragile]{Interlude: environnements}
\begin{itemize}
\item \bftt{equation} est un \emph{environnement} --- un contexte.
\item Une commande peut produire des sorties différentes selon le contexte.
\begin{exampletwouptiny}
On peut écrire
$ \Omega = \sum_{k=1}^{n} \omega_k $
dans le texte courant, mais on peut
aussi écrire
\begin{equation}
  \Omega = \sum_{k=1}^{n} \omega_k
\end{equation}
dans une formule en vedette.
\end{exampletwouptiny}
\vskip 2ex
\item Notez comment le symbole $\Sigma$ est plus grand à l'intérieur de l'environnement \bftt{equation}, et comment les indices et exposants changent de position, malgré le fait que nous avons utilisé les mêmes commandes.
\vskip 1em
{\scriptsize En fait nous pourrions aussi écrire \cmdbegin{math}\bftt{...}\cmdend{math} à la place de \rlap{\bftt{\$...\$}.}

}
\end{itemize}
\end{frame}

%%%%%%%%%%%%%%%%%%%%%%%%%%%%%%%%%%%%%%%%%%%%%%%%%%%%%%%%%%%%%%%%%%%%%%%%%%%%%%%
%%%%%%%%%%%%%%%%%%%%%%%%%%%%%%%%%%%%%%%%%%%%%%%%%%%%%%%%%%%%%%%%%%%%%%%%%%%%%%%
%%%%%%%%%%%%%%%%%%%%%%%%%%%%%%%%%%%%%%%%%%%%%%%%%%%%%%%%%%%%%%%%%%%%%%%%%%%%%%%
\begin{frame}[fragile]{Interlude: environnements}
\begin{itemize}
\item Les commandes \cmdbs{begin} et \cmdbs{end} sont utilisées pour créer des environnements.
\vskip 2ex

\item Les environnements \bftt{itemize} et \bftt{enumerate} créent des listes.
\begin{exampletwouptiny}
\begin{itemize} % pour avoir des puces
\item Biscuits
\item Thé
\end{itemize}

\begin{enumerate} % énumération
\item Biscuits
\item Thé
\end{enumerate}
\end{exampletwouptiny}
\end{itemize}
\end{frame}

%%%%%%%%%%%%%%%%%%%%%%%%%%%%%%%%%%%%%%%%%%%%%%%%%%%%%%%%%%%%%%%%%%%%%%%%%%%%%%%
%%%%%%%%%%%%%%%%%%%%%%%%%%%%%%%%%%%%%%%%%%%%%%%%%%%%%%%%%%%%%%%%%%%%%%%%%%%%%%%
%%%%%%%%%%%%%%%%%%%%%%%%%%%%%%%%%%%%%%%%%%%%%%%%%%%%%%%%%%%%%%%%%%%%%%%%%%%%%%%
\begin{frame}[fragile]{Interlude: packages}

\begin{itemize}
\item Toutes les commandes et tous les environnements vus jusqu'à maintenant font partie du noyau central de \LaTeX.

\item Les \emph{packages} sont des bibliothèques de commandes et d'environnements supplémentaires. Il y a des milliers de packages libres qui sont disponibles.

\item Il faut charger les packages que nous souhaitons utiliser par la commande
\cmdbs{usepackage} à placer dans le \emph{préambule}.

\item Exemple : le package \bftt{amsmath} de la Société mathématique américaine.
\begin{minted}[fontsize=\small,frame=single]{latex}
\documentclass{article}
\usepackage{amsmath} % préambule
\begin{document}
% ici vous pouvez utiliser les commandes amsmath...
\end{document}
\end{minted}
\end{itemize}
\end{frame}

%%%%%%%%%%%%%%%%%%%%%%%%%%%%%%%%%%%%%%%%%%%%%%%%%%%%%%%%%%%%%%%%%%%%%%%%%%%%%%%
%%%%%%%%%%%%%%%%%%%%%%%%%%%%%%%%%%%%%%%%%%%%%%%%%%%%%%%%%%%%%%%%%%%%%%%%%%%%%%%
%%%%%%%%%%%%%%%%%%%%%%%%%%%%%%%%%%%%%%%%%%%%%%%%%%%%%%%%%%%%%%%%%%%%%%%%%%%%%%%
\begin{frame}[fragile]{\insertsubsection{}: exemples avec \bftt{amsmath}}
\begin{itemize}
\item Use \bftt{equation*} (\og équation étoilée\fg) pour les équations non numérotées.
\begin{exampletwouptiny}
\begin{equation*}
  \Omega = \sum_{k=1}^{n} \omega_k
\end{equation*}
\end{exampletwouptiny}
\item \LaTeX{} traite les lettres adjacentes comme s'il s'agissait de produits de variables, ce qui n'est pas toujours ce que vous souhaitez. Le package \bftt{amsmath} definit des commandes pour la plupart des opérateurs mathématiques.
\begin{exampletwouptiny}
\begin{equation*} % pas bon !
 min_{x,y} (1-x)^2 + 100(y-x^2)^2
\end{equation*}
\begin{equation*} % bon !
\min_{x,y}{(1-x)^2 + 100(y-x^2)^2}
\end{equation*}
\end{exampletwouptiny}
\item Vous pouvez utiliser \cmdbs{operatorname} pour les autres.
\begin{exampletwouptiny}
\begin{equation*}
\beta_i =
\frac{\operatorname{Cov}(R_i, R_m)}
     {\operatorname{Var}(R_m)}
\end{equation*}
\end{exampletwouptiny}
\end{itemize}
\end{frame}

%%%%%%%%%%%%%%%%%%%%%%%%%%%%%%%%%%%%%%%%%%%%%%%%%%%%%%%%%%%%%%%%%%%%%%%%%%%%%%%
%%%%%%%%%%%%%%%%%%%%%%%%%%%%%%%%%%%%%%%%%%%%%%%%%%%%%%%%%%%%%%%%%%%%%%%%%%%%%%%
%%%%%%%%%%%%%%%%%%%%%%%%%%%%%%%%%%%%%%%%%%%%%%%%%%%%%%%%%%%%%%%%%%%%%%%%%%%%%%%
\begin{frame}[fragile]{\insertsubsection{}: exemples avec \bftt{amsmath}}
\begin{itemize}{\small
\item Aligner une série d'équations au signe égal
\begin{align*}
(x+1)^3 &= (x+1)(x+1)(x+1) \\
        &= (x+1)(x^2 + 2x + 1) \\
        &= x^3 + 3x^2 + 3x + 1
\end{align*}
avec l'environnement \bftt{align*}.

% pour une raison inconnue ceci n'est pas compatible avec l'environnement twoup
\begin{minted}[fontsize=\small,frame=single]{latex}
\begin{align*}
(x+1)^3 &= (x+1)(x+1)(x+1) \\
        &= (x+1)(x^2 + 2x + 1) \\
        &= x^3 + 3x^2 + 3x + 1
\end{align*}
\end{minted}
\item Une esperluette \keystrokebftt{\&} sépare la colonne de gauche (avant le
$=$) de la colonne de droite (après le $=$).
\item Un double antislash \keystrokebftt{\bs}\keystrokebftt{\bs} passe à la ligne.
}\end{itemize}
\end{frame}


%%%%%%%%%%%%%%%%%%%%%%%%%%%%%%%%%%%%%%%%%%%%%%%%%%%%%%%%%%%%%%%%%%%%%%%%%%%%%%%
%%%%%%%%%%%%%%%%%%%%%%%%%%%%%%%%%%%%%%%%%%%%%%%%%%%%%%%%%%%%%%%%%%%%%%%%%%%%%%%
%%%%%%%%%%%%%%%%%%%%%%%%%%%%%%%%%%%%%%%%%%%%%%%%%%%%%%%%%%%%%%%%%%%%%%%%%%%%%%%
\begin{frame}[fragile]{Exercice de composition 2}

\begin{block}{Composez ceci sous \LaTeX:}
Soit $X_1, X_2, \ldots, X_n$ une suite de variables aléatoires
indépendantes et identiquement distribuées avec 
$\operatorname{E}[X_i] = \mu$ et $\operatorname{Var}[X_i] = \sigma^2 < \infty$, 
et soit
\begin{equation*}
S_n = \frac{1}{n}\sum_{i}^{n} X_i
\end{equation*}
leur moyenne. Alors quand $n$ tend vers l'infini,
la racine carrée des variables aléatoires $\sqrt{n}(S_n - \mu)$
converge en distribution vers la loi normale $N(0, \sigma^2)$.
\end{block}
\vskip 2ex
\begin{center}
\fbox{\href{\wlnewdoc{basics-exercise-2.tex}}{%
Cliquer ici pour l'ouvrir sous \wllogo{}}}
\end{center}
\begin{itemize}
\item Tuyau : la commande pour obtenir le symbole $\infty$ est \cmdbs{infty}.
\item Après avoir essayé,
\fbox{\href{\wlnewdoc{basics-exercise-2-solution.tex}}{%
cliquez ici pour voir ma solution}}.
\end{itemize}
\end{frame}

%%%%%%%%%%%%%%%%%%%%%%%%%%%%%%%%%%%%%%%%%%%%%%%%%%%%%%%%%%%%%%%%%%%%%%%%%%%%%%%
%%%%%%%%%%%%%%%%%%%%%%%%%%%%%%%%%%%%%%%%%%%%%%%%%%%%%%%%%%%%%%%%%%%%%%%%%%%%%%%
%%%%%%%%%%%%%%%%%%%%%%%%%%%%%%%%%%%%%%%%%%%%%%%%%%%%%%%%%%%%%%%%%%%%%%%%%%%%%%%
\begin{frame}{Fin de la première partie}
\begin{itemize}
\item Félicitations ! Vous avez appris comment...
\begin{itemize}
\item Composer du texte en \LaTeX.
\item Utiliser un tas de commandes différentes.
\item Gérer les erreurs quand elles surviennent.
\item Composer de très belles formules mathématiques.
\item Utiliser différents environnements.
\item Charger des packages.
\end{itemize}
\item C'est incroyable !
\item Dans la partie~2, nous verrons comment utiliser \LaTeX{} pour produire des documents structurés avec des sections, des références croisées, des figures, des tables et des bibliographies. À bientôt !
\end{itemize}
\end{frame}

\end{document}
