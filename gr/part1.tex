\documentclass{beamer}

%
% Common preamble for all three parts.
%

\usepackage[english]{babel}
\usepackage{amsmath}
\usepackage{color}
\usepackage{minted}
\usepackage{hyperref}
\usepackage{multicol}
\usepackage{tabularx}
\usepackage{tikz}

% only inline todonotes work
\usepackage{xkeyval}
\usepackage[textsize=small]{todonotes}
\presetkeys{todonotes}{inline}{}

\usetikzlibrary{shapes,arrows,positioning,shadows}

% no nav buttons
\usenavigationsymbolstemplate{}

\newcommand{\bftt}[1]{\textbf{\texttt{#1}}}
\newcommand{\comment}[1]{{\color[HTML]{008080}\textit{\textbf{\texttt{#1}}}}}
\newcommand{\cmd}[1]{{\color[HTML]{008000}\bftt{#1}}}
\newcommand{\bs}{\char`\\}
\newcommand{\cmdbs}[1]{\cmd{\bs#1}}
\newcommand{\lcb}{\char '173}
\newcommand{\rcb}{\char '175}
\newcommand{\cmdbegin}[1]{\cmdbs{begin\lcb}\bftt{#1}\cmd{\rcb}}
\newcommand{\cmdend}[1]{\cmdbs{end\lcb}\bftt{#1}\cmd{\rcb}}

\newcommand{\wllogo}{\textbf{write\textrm{\LaTeX}}}

% this is where the example source files are loaded from
% do not include a trailing slash
\newcommand{\fileuri}{https://raw.github.com/jdleesmiller/latex-course/master/en}

\newcommand{\wlserver}{https://www.writelatex.com}
\newcommand{\wlnewdoc}[1]{\wlserver/docs?snip\_uri=\fileuri/#1\&splash=none}

\def\tikzname{Ti\emph{k}Z}

% from http://tex.stackexchange.com/questions/5226/keyboard-font-for-latex
\newcommand*\keystroke[1]{%
  \tikz[baseline=(key.base)]
    \node[%
      draw,
      fill=white,
      drop shadow={shadow xshift=0.25ex,shadow yshift=-0.25ex,fill=black,opacity=0.75},
      rectangle,
      rounded corners=2pt,
      inner sep=1pt,
      line width=0.5pt,
      font=\scriptsize\sffamily
    ](key) {#1\strut}
  ;
}
\newcommand{\keystrokebftt}[1]{\keystroke{\bftt{#1}}}

% stolen from minted.dtx
\newenvironment{exampletwoup}
  {\VerbatimEnvironment
   \begin{VerbatimOut}{example.out}}
  {\end{VerbatimOut}
   \setlength{\parindent}{0pt}
   \fbox{\begin{tabular}{l|l}
   \begin{minipage}{0.55\linewidth}
     \inputminted[fontsize=\small,resetmargins]{latex}{example.out}
   \end{minipage} &
   \begin{minipage}{0.35\linewidth}
     \input{example.out}
   \end{minipage}
   \end{tabular}}}

\newenvironment{exampletwouptiny}
  {\VerbatimEnvironment
   \begin{VerbatimOut}{example.out}}
  {\end{VerbatimOut}
   \setlength{\parindent}{0pt}
   \fbox{\begin{tabular}{l|l}
   \begin{minipage}{0.55\linewidth}
     \inputminted[fontsize=\scriptsize,resetmargins]{latex}{example.out}
   \end{minipage} &
   \begin{minipage}{0.35\linewidth}
     \setlength{\parskip}{6pt plus 1pt minus 1pt}%
     \raggedright\scriptsize\input{example.out}
   \end{minipage}
   \end{tabular}}}

\newenvironment{exampletwouptinynoframe}
  {\VerbatimEnvironment
   \begin{VerbatimOut}{example.out}}
  {\end{VerbatimOut}
   \setlength{\parindent}{0pt}
   \begin{tabular}{l|l}
   \begin{minipage}{0.55\linewidth}
     \inputminted[fontsize=\scriptsize,resetmargins]{latex}{example.out}
   \end{minipage} &
   \begin{minipage}{0.35\linewidth}
     \setlength{\parskip}{6pt plus 1pt minus 1pt}%
     \raggedright\scriptsize\input{example.out}
   \end{minipage}
   \end{tabular}}

\title{An Interactive Introduction to \LaTeX}
\author{Dr John D. Lees-Miller}
\titlegraphic{%
\includegraphics[page=4]{wllogo-series}\\[1em]
\includegraphics[height=24pt]{UoB-logo}
\qquad
\includegraphics[height=24pt]{setsquared_supported}
}



\subtitle{Μέρος 1: Βασικές Γνώσεις}

\begin{document}
\gr
%%%%%%%%%%%%%%%%%%%%%%%%%%%%%%%%%%%%%%%%%%%%%%%%%%%%%%%%%%%%%%%%%%%%%%%%%%%%%%%
%%%%%%%%%%%%%%%%%%%%%%%%%%%%%%%%%%%%%%%%%%%%%%%%%%%%%%%%%%%%%%%%%%%%%%%%%%%%%%%
%%%%%%%%%%%%%%%%%%%%%%%%%%%%%%%%%%%%%%%%%%%%%%%%%%%%%%%%%%%%%%%%%%%%%%%%%%%%%%%
\begin{frame}
\titlepage
\end{frame}
\gr
%%%%%%%%%%%%%%%%%%%%%%%%%%%%%%%%%%%%%%%%%%%%%%%%%%%%%%%%%%%%%%%%%%%%%%%%%%%%%%%
%%%%%%%%%%%%%%%%%%%%%%%%%%%%%%%%%%%%%%%%%%%%%%%%%%%%%%%%%%%%%%%%%%%%%%%%%%%%%%%
%%%%%%%%%%%%%%%%%%%%%%%%%%%%%%%%%%%%%%%%%%%%%%%%%%%%%%%%%%%%%%%%%%%%%%%%%%%%%%%
\begin{frame}{Γιατί \LaTeX{}?}
\begin{itemize}
\item Φτιάχνει όμορφα έγγραφα
\begin{itemize}
\item Ειδικά μαθηματικά έγγραφα
\end{itemize}
%
\item Δημιουργήθηκε από επιστήμονες, για επιστήμονες
\begin{itemize}
\item Μια μεγάλη και ενεργή κοινότητα
\end{itemize}
%
\item Είναι ισχυρό εργαλείο--- μπορείτε να το επεκτείνετε
\begin{itemize}
\item Πακέτα για έγγραφα, παρουσιάσεις, υπολογιστικά φύλλα, \ldots
\end{itemize}
\end{itemize}
\end{frame}

%%%%%%%%%%%%%%%%%%%%%%%%%%%%%%%%%%%%%%%%%%%%%%%%%%%%%%%%%%%%%%%%%%%%%%%%%%%%%%%
%%%%%%%%%%%%%%%%%%%%%%%%%%%%%%%%%%%%%%%%%%%%%%%%%%%%%%%%%%%%%%%%%%%%%%%%%%%%%%%
%%%%%%%%%%%%%%%%%%%%%%%%%%%%%%%%%%%%%%%%%%%%%%%%%%%%%%%%%%%%%%%%%%%%%%%%%%%%%%%
\begin{frame}[fragile]{Πώς λειτουργεί?}
\begin{itemize}
\item Γράφεις το έγγραφό σου σε \texttt{απλό κείμενο} με \cmd{εντολές} που περιγράφουν τη δομή και τη σημασία του.
\item To \texttt{\en latex\gr} επεξεργάζεται το κείμενο και τις εντολές σας για να παράγει ένα
όμορφα διαμορφωμένο έγγραφο.
\end{itemize}
\vskip 2ex
\begin{center}
\en
\begin{minted}[frame=single]{latex}
The rain in Spain falls \emph{mainly} on the plain.
\end{minted}
\vskip 2ex
\tikz\node[single arrow,fill=gray,font=\ttfamily\bfseries,%
  rotate=270,xshift=-1em]{\textit{latex} };
\vskip 2ex
\fbox{The rain in Spain falls \emph{mainly} on the plain.}
\end{center}
\end{frame}
\gr
%%%%%%%%%%%%%%%%%%%%%%%%%%%%%%%%%%%%%%%%%%%%%%%%%%%%%%%%%%%%%%%%%%%%%%%%%%%%%%%
%%%%%%%%%%%%%%%%%%%%%%%%%%%%%%%%%%%%%%%%%%%%%%%%%%%%%%%%%%%%%%%%%%%%%%%%%%%%%%%
%%%%%%%%%%%%%%%%%%%%%%%%%%%%%%%%%%%%%%%%%%%%%%%%%%%%%%%%%%%%%%%%%%%%%%%%%%%%%%%

\begin{frame}[fragile]{Περισσότερα παραδείγματα εντολών και η έξοδος τους\ldots}
\en
\begin{exampletwoup}
\begin{itemize}
\item Tea
\item Milk
\item Biscuits
\end{itemize}
\end{exampletwoup}
\vskip 2ex
\begin{exampletwoup}
\begin{figure}
\includegraphics{gerbil}
\end{figure}
\end{exampletwoup}
\vskip 2ex
\begin{exampletwoup}
\begin{equation}
\alpha + \beta + 1
\end{equation}
\end{exampletwoup}

\tiny{Image license: \href{https://pixabay.com/en/animal-apple-attractive-beautiful-1239390/}{CC0}}
\end{frame}

%%%%%%%%%%%%%%%%%%%%%%%%%%%%%%%%%%%%%%%%%%%%%%%%%%%%%%%%%%%%%%%%%%%%%%%%%%%%%%%
%%%%%%%%%%%%%%%%%%%%%%%%%%%%%%%%%%%%%%%%%%%%%%%%%%%%%%%%%%%%%%%%%%%%%%%%%%%%%%%
%%%%%%%%%%%%%%%%%%%%%%%%%%%%%%%%%%%%%%%%%%%%%%%%%%%%%%%%%%%%%%%%%%%%%%%%%%%%%%%
\begin{frame}[fragile]{Προσαρμογή μορφής}

\begin{itemize}
\item Χρησιμοποιήστε εντολές για να περιγράψετε <<τι είναι>>, όχι <<πώς φαίνεται>>
\item Εστιάστε στο περιεχόμενό σας.
\item Αφήστε το \LaTeX{} να κάνει τη δουλειά του.
\end{itemize}
\end{frame}

%%%%%%%%%%%%%%%%%%%%%%%%%%%%%%%%%%%%%%%%%%%%%%%%%%%%%%%%%%%%%%%%%%%%%%%%%%%%%%%
%%%%%%%%%%%%%%%%%%%%%%%%%%%%%%%%%%%%%%%%%%%%%%%%%%%%%%%%%%%%%%%%%%%%%%%%%%%%%%%
%%%%%%%%%%%%%%%%%%%%%%%%%%%%%%%%%%%%%%%%%%%%%%%%%%%%%%%%%%%%%%%%%%%%%%%%%%%%%%%
\section{The Basics}

%%%%%%%%%%%%%%%%%%%%%%%%%%%%%%%%%%%%%%%%%%%%%%%%%%%%%%%%%%%%%%%%%%%%%%%%%%%%%%%
%%%%%%%%%%%%%%%%%%%%%%%%%%%%%%%%%%%%%%%%%%%%%%%%%%%%%%%%%%%%%%%%%%%%%%%%%%%%%%%
%%%%%%%%%%%%%%%%%%%%%%%%%%%%%%%%%%%%%%%%%%%%%%%%%%%%%%%%%%%%%%%%%%%%%%%%%%%%%%%
\subsection{Ξεκινώντας}
\begin{frame}[fragile]{\insertsubsection}
\begin{itemize}
\item Ένα μικρό έγγραφο \LaTeX{}:\en
\inputminted[frame=single]{latex}{basics.tex}\gr
\item Οι εντολές ξεκινούν με \en \emph{backslash} \keystrokebftt{\bs}.\gr
\item 
Κάθε έγγραφο ξεκινά με μια εντολή \en \cmdbs{documentclass}\gr.
\item Η \emph{εντολή} σε σγουρά άγκιστρα \keystrokebftt{\{} \keystrokebftt{\}} λέει στο \LaTeX{} τι είδους έγγραφο δημιουργούμε: ένα \bftt{άρθρο}.
\item Ένα σύμβολο ποσοστού \keystrokebftt{\%} ξεκινά ένα \emph{σχόλιο} --- \LaTeX{}
θα αγνοήσει την υπόλοιπη γραμμή.
\end{itemize}
\end{frame}

%%%%%%%%%%%%%%%%%%%%%%%%%%%%%%%%%%%%%%%%%%%%%%%%%%%%%%%%%%%%%%%%%%%%%%%%%%%%%%%
%%%%%%%%%%%%%%%%%%%%%%%%%%%%%%%%%%%%%%%%%%%%%%%%%%%%%%%%%%%%%%%%%%%%%%%%%%%%%%%
%%%%%%%%%%%%%%%%%%%%%%%%%%%%%%%%%%%%%%%%%%%%%%%%%%%%%%%%%%%%%%%%%%%%%%%%%%%%%%%
\begin{frame}[fragile]{\insertsubsection{} με το \en \wllogo \gr}
\begin{itemize}
\item Το $Overleaf$ είναι ένας ιστότοπος για τη σύνταξη εγγράφων στο \LaTeX.
\item «Μετατρέπει» αυτόματα το \LaTeX{} σας για να σας δείξει τα αποτελέσματα.
\vskip 2em
\begin{center}
\fbox{\href{\wlnewdoc{basics.tex}}{%
Κάντε κλικ εδώ για να ανοίξετε το παράδειγμα  στο \en \wllogo{}\gr}}
\\[1ex]\scriptsize{}
Για καλύτερα αποτελέσματα, χρησιμοποιήστε \en \href{http://www.google.com/chrome}{Google Chrome} \gr  ή \en \href{http://www.mozilla.org/en-US/firefox/new/}{FireFox}.\gr
\end{center}
\vskip 2ex
\item Καθώς προχωράμε στις παρακάτω διαφάνειες, δοκιμάστε τα παραδείγματα πληκτρολογώντας τα
στο παράδειγμα εγγράφου στο $Overleaf$.
\item \textbf{Όχι πραγματικά, θα πρέπει να τα δοκιμάσετε όσο προχωράμε!}
\end{itemize}
\end{frame}

%%%%%%%%%%%%%%%%%%%%%%%%%%%%%%%%%%%%%%%%%%%%%%%%%%%%%%%%%%%%%%%%%%%%%%%%%%%%%%%
%%%%%%%%%%%%%%%%%%%%%%%%%%%%%%%%%%%%%%%%%%%%%%%%%%%%%%%%%%%%%%%%%%%%%%%%%%%%%%%
%%%%%%%%%%%%%%%%%%%%%%%%%%%%%%%%%%%%%%%%%%%%%%%%%%%%%%%%%%%%%%%%%%%%%%%%%%%%%%%
\subsection{Πληκτρολόγηση κειμένου}
\begin{frame}[fragile]{\insertsubsection{}}
\small
\begin{itemize}
\item Πληκτρολογήστε το κείμενό σας μεταξύ \en \cmdbegin{document} \gr και \en \cmdend{document} \gr.
\item Ως επί το πλείστον, μπορείτε απλώς να πληκτρολογήσετε το κείμενό σας κανονικά.
\en
\begin{exampletwouptiny}
Words are separated by one or more
spaces.

Paragraphs are separated by one
or more blank lines.

\end{exampletwouptiny}
\gr
\item Ο χώρος στο αρχείο προέλευσης συμπτύσσεται στην έξοδο.
\en
\begin{exampletwouptiny}
The   rain       in Spain
falls mainly on the plain.
\end{exampletwouptiny}
\end{itemize}
\end{frame}
\gr
%%%%%%%%%%%%%%%%%%%%%%%%%%%%%%%%%%%%%%%%%%%%%%%%%%%%%%%%%%%%%%%%%%%%%%%%%%%%%%%
%%%%%%%%%%%%%%%%%%%%%%%%%%%%%%%%%%%%%%%%%%%%%%%%%%%%%%%%%%%%%%%%%%%%%%%%%%%%%%%
%%%%%%%%%%%%%%%%%%%%%%%%%%%%%%%%%%%%%%%%%%%%%%%%%%%%%%%%%%%%%%%%%%%%%%%%%%%%%%%
\begin{frame}[fragile]{\insertsubsection{}: }
\small
\begin{itemize}
\item Τα εισαγωγικά είναι λίγο δύσκολα:\\
χρησιμοποιήστε ένα $backtick$ \keystroke{\`{}} στα αριστερά και μια απόστροφο \keystroke{\'{}} στα δεξιά.
\en
\begin{exampletwouptiny}
Single quotes: `text'.

Double quotes: ``text''.
\end{exampletwouptiny}
\gr
\item Ορισμένοι κοινοί χαρακτήρες έχουν ειδική σημασία στο \LaTeX:\\[1ex]
\en
\begin{tabular}{cl}
\keystrokebftt{\%} & percent sign              \\
\keystrokebftt{\#} & hash (pound / sharp) sign \\
\keystrokebftt{\&} & ampersand                 \\
\keystrokebftt{\$} & dollar sign               \\
\end{tabular}
\gr
\item Αν απλώς πληκτρολογήσετε αυτά, θα λάβετε ένα σφάλμα. Αν θέλετε να εμφανιστεί ένα
στην έξοδο, πρέπει να το \emph{εισάγετε} προηγουμένως με μια ανάστροφη κάθετο.
\en
\begin{exampletwoup}
\$\%\&\#!
\end{exampletwoup}
\end{itemize}
\end{frame}
\gr
%%%%%%%%%%%%%%%%%%%%%%%%%%%%%%%%%%%%%%%%%%%%%%%%%%%%%%%%%%%%%%%%%%%%%%%%%%%%%%%
%%%%%%%%%%%%%%%%%%%%%%%%%%%%%%%%%%%%%%%%%%%%%%%%%%%%%%%%%%%%%%%%%%%%%%%%%%%%%%%
%%%%%%%%%%%%%%%%%%%%%%%%%%%%%%%%%%%%%%%%%%%%%%%%%%%%%%%%%%%%%%%%%%%%%%%%%%%%%%%
\begin{frame}[fragile]{Σφάλματα χειρισμού}
\begin{itemize}
\item Το \LaTeX{} μπορεί να μπερδευτεί όταν προσπαθεί να μεταγλωττίσει το έγγραφό σας. Αν
το κάνει, σταματά με ένα σφάλμα, το οποίο πρέπει να διορθώσετε πριν εμφανιστεί
οποιαδήποτε έξοδος.
\item Για παράδειγμα, εάν γράψετε λάθος \en \cmdbs{emph} \gr ως \en \cmdbs{meph} \gr, το \LaTeX{} θα
σταματήστε με ένα σφάλμα \en ``undefined control sequence''\gr, επειδή το $"meph"$ δεν είναι
μια από τις εντολές που ξέρει.
\end{itemize}
\begin{block}{Συμβουλές για σφάλματα}
\begin{enumerate}
\item Μην πανικοβάλλεστε! Συμβαίνουν λάθη.
\item Διορθώστε τα αμέσως μόλις προκύψουν --- εάν αυτό που μόλις πληκτρολογήσατε προκάλεσε σφάλμα,
μπορείτε να ξεκινήσετε τον εντοπισμό σφαλμάτων από εκεί.
\item Εάν υπάρχουν πολλά σφάλματα, ξεκινήστε με το πρώτο --- η αιτία μπορεί
ακόμη και να είναι πάνω από αυτό.
\end{enumerate}
\end{block}
\end{frame}

%%%%%%%%%%%%%%%%%%%%%%%%%%%%%%%%%%%%%%%%%%%%%%%%%%%%%%%%%%%%%%%%%%%%%%%%%%%%%%%
%%%%%%%%%%%%%%%%%%%%%%%%%%%%%%%%%%%%%%%%%%%%%%%%%%%%%%%%%%%%%%%%%%%%%%%%%%%%%%%
%%%%%%%%%%%%%%%%%%%%%%%%%%%%%%%%%%%%%%%%%%%%%%%%%%%%%%%%%%%%%%%%%%%%%%%%%%%%%%%
\begin{frame}[fragile]{Άσκηση στοιχειοθέτησης 1}

\begin{block}{Πληκτρολόγηση σε \LaTeX:\en
\footnote{\url{http://en.wikipedia.org/wiki/Economy_of_the_United_States}}}\gr
\gr
Τον Μάρτιο του 2006, το Κογκρέσο αύξησε αυτό το ανώτατο όριο κατά  \$0,79 τρισεκατομμύρια  σε \$8,97 τρισεκατομμύρια, που είναι περίπου το 68\% του ΑΕΠ. Από τις 4 Οκτωβρίου 2008, ο «Νόμος Έκτακτης Οικονομικής Σταθεροποίησης του 2008» αύξησε το τρέχον ανώτατο όριο χρέους στα \$11,3 τρις.

\end{block}
\vskip 2ex
\begin{center}
\fbox{\href{\wlnewdoc{basics-exercise-1.tex}}{%
Κάντε κλικ για να ανοίξετε αυτήν την άσκηση στο \en \wllogo{}}}\gr
\end{center}

\begin{itemize}
\item Συμβουλή: προσέξτε τους χαρακτήρες με ιδιαίτερη σημασία!
\item Μόλις δοκιμάσετε,
\fbox{\href{\wlnewdoc{basics-exercise-1-solution.tex}}{%
κάντε κλικ εδώ για να δείτε τη λύση μου}}.
\end{itemize}
\end{frame}

%%%%%%%%%%%%%%%%%%%%%%%%%%%%%%%%%%%%%%%%%%%%%%%%%%%%%%%%%%%%%%%%%%%%%%%%%%%%%%%
%%%%%%%%%%%%%%%%%%%%%%%%%%%%%%%%%%%%%%%%%%%%%%%%%%%%%%%%%%%%%%%%%%%%%%%%%%%%%%%
%%%%%%%%%%%%%%%%%%%%%%%%%%%%%%%%%%%%%%%%%%%%%%%%%%%%%%%%%%%%%%%%%%%%%%%%%%%%%%%
\subsection{Πληκτρολόγηση Μαθηματικών}
\begin{frame}[fragile]{\insertsubsection{}: Ένδειξη Δολάριου}
\begin{itemize}
\item Γιατί το σύμβολο του δολαρίου $\keystrokebftt{\$}$ είναι ιδιαίτερο$;$ Το χρησιμοποιούμε για να γράψουμε τα μαθηματικά σε κείμενο.\\[1ex]
\en
\begin{exampletwouptiny}
% not so good:
Let a and b be distinct positive
integers, and let c = a - b + 1.

% much better:
Let $a$ and $b$ be distinct positive
integers, and let $c = a - b + 1$.
\end{exampletwouptiny}
\gr
\item Χρησιμοποιείτε πάντα σύμβολα δολαρίου σε ζευγάρια --- ένα για να ξεκινήσετε τα μαθηματικά και ένα
να τα τελειώσετε.
\item Το \LaTeX{} χειρίζεται αυτόματα την απόσταση, αγνοεί τα κενά σας.
\en
\begin{exampletwouptiny}
Let $y=mx+b$ be \ldots

Let $y = m x + b$ be \ldots
\end{exampletwouptiny}
\end{itemize}
\end{frame}
\gr

%%%%%%%%%%%%%%%%%%%%%%%%%%%%%%%%%%%%%%%%%%%%%%%%%%%%%%%%%%%%%%%%%%%%%%%%%%%%%%%
%%%%%%%%%%%%%%%%%%%%%%%%%%%%%%%%%%%%%%%%%%%%%%%%%%%%%%%%%%%%%%%%%%%%%%%%%%%%%%%
%%%%%%%%%%%%%%%%%%%%%%%%%%%%%%%%%%%%%%%%%%%%%%%%%%%%%%%%%%%%%%%%%%%%%%%%%%%%%%%
\begin{frame}[fragile]{\insertsubsection{}: Χρήση Συμβόλων}
\begin{itemize}
\item Χρησιμοποιήστε το \en caret \gr $\keystrokebftt{\^}$ για τους εκθέτες και την υπογράμμιση $\keystrokebftt{\_}$ για τους δείκτες
\en
\begin{exampletwouptiny}
$y = c_2 x^2 + c_1 x + c_0$
\end{exampletwouptiny}
\gr
\vskip 2ex

\item Χρησιμοποιήστε σγουρά άγκιστρα \keystrokebftt{\{} \keystrokebftt{\}} για ομαδοποίηση σε εκθέτες και δείκτες.
\en
\begin{exampletwouptiny}
$F_n = F_n-1 + F_n-2$     % oops!

$F_n = F_{n-1} + F_{n-2}$ % ok!
\end{exampletwouptiny}
\gr
\vskip 2ex

\item Υπάρχουν εντολές για ελληνικά γράμματα και κοινή σημειογραφία.
\en
\begin{exampletwouptiny}
$\mu = A e^{Q/RT}$

$\Omega = \sum_{k=1}^{n} \omega_k$
\end{exampletwouptiny}
\end{itemize}
\end{frame}
\gr
%%%%%%%%%%%%%%%%%%%%%%%%%%%%%%%%%%%%%%%%%%%%%%%%%%%%%%%%%%%%%%%%%%%%%%%%%%%%%%%
%%%%%%%%%%%%%%%%%%%%%%%%%%%%%%%%%%%%%%%%%%%%%%%%%%%%%%%%%%%%%%%%%%%%%%%%%%%%%%%
%%%%%%%%%%%%%%%%%%%%%%%%%%%%%%%%%%%%%%%%%%%%%%%%%%%%%%%%%%%%%%%%%%%%%%%%%%%%%%%
\begin{frame}[fragile]{\insertsubsection{}: Εμφάνιση εξισώσεων}
\begin{itemize}
\item 
Αν η εξίσωση είναι μεγάλη και τρομακτική, \emph{εμφανίσε} την στη δική της γραμμή χρησιμοποιώντας
\en \cmdbegin{equation} \gr και \en \cmdend{equation} \gr.\\[2ex]
\en
\begin{exampletwouptiny}
The roots of a quadratic equation
are given by
\begin{equation}
x = \frac{-b \pm \sqrt{b^2 - 4ac}}
         {2a}
\end{equation}
where $a$, $b$ and $c$ are \ldots
\end{exampletwouptiny}
\gr
\vskip 1em
{\scriptsize Προσοχή: Το \LaTeX{} συνήθως αγνοεί τα κενά σας στα μαθηματικά, αλλά
δεν μπορεί να διαχειριστεί κενές γραμμές στις εξισώσεις --- μην βάζετε κενές γραμμές στο δικό σας
μαθηματικά.}
\end{itemize}
\end{frame}

%%%%%%%%%%%%%%%%%%%%%%%%%%%%%%%%%%%%%%%%%%%%%%%%%%%%%%%%%%%%%%%%%%%%%%%%%%%%%%%
%%%%%%%%%%%%%%%%%%%%%%%%%%%%%%%%%%%%%%%%%%%%%%%%%%%%%%%%%%%%%%%%%%%%%%%%%%%%%%%
%%%%%%%%%%%%%%%%%%%%%%%%%%%%%%%%%%%%%%%%%%%%%%%%%%%%%%%%%%%%%%%%%%%%%%%%%%%%%%%
\begin{frame}[fragile]{Ενδιάμεσα: Περιβάλλοντα}
\begin{itemize}
\item Η \bftt{εξίσωση} είναι ένα \emph{περιβάλλον} --- ένα πλαίσιο.
\item Μια εντολή μπορεί να παράγει διαφορετικά αποτελέσματα σε διαφορετικά περιβάλλοντα.
\en
\begin{exampletwouptiny}
We can write
$ \Omega = \sum_{k=1}^{n} \omega_k $
in text, or we can write
\begin{equation}
  \Omega = \sum_{k=1}^{n} \omega_k
\end{equation}
to display it.

\end{exampletwouptiny}
\gr
\vskip 2ex
\item Σημειώστε πώς το $\Sigma$ είναι μεγαλύτερο στο περιβάλλον $\bftt{equation}$ και
πώς αλλάζουν θέση οι δείκτες και οι εκθέτες, παρόλο που χρησιμοποιήσαμε το
ίδιες εντολές.
\vskip 1em
{\scriptsize Στην πραγματικότητα, θα μπορούσαμε να έχουμε γράψει \en \bftt{\$...\$} \gr ως
\en \cmdbegin{math}\bftt{...}\cmdend{math}\gr.}
\end{itemize}
\end{frame}

%%%%%%%%%%%%%%%%%%%%%%%%%%%%%%%%%%%%%%%%%%%%%%%%%%%%%%%%%%%%%%%%%%%%%%%%%%%%%%%
%%%%%%%%%%%%%%%%%%%%%%%%%%%%%%%%%%%%%%%%%%%%%%%%%%%%%%%%%%%%%%%%%%%%%%%%%%%%%%%
%%%%%%%%%%%%%%%%%%%%%%%%%%%%%%%%%%%%%%%%%%%%%%%%%%%%%%%%%%%%%%%%%%%%%%%%%%%%%%%
\begin{frame}[fragile]{Ενδιάμεσα: Περιβάλλοντα}
\begin{itemize}
\item Οι εντολές \en \cmdbs{begin} \gr και \en \cmdbs{end} \gr χρησιμοποιούνται για τη δημιουργία πολλών
διαφορετικών περιβάλλοντων.
\vskip 2ex

\item Το περιβάλλον \en \bftt{itemize} \gr και \en \bftt{enumerate} \gr δημιουργεί λίστες.
\en
\begin{exampletwouptiny}
\begin{itemize} % for bullet points
\item Biscuits
\item Tea
\end{itemize}

\begin{enumerate} % for numbers
\item Biscuits
\item Tea
\end{enumerate}
\end{exampletwouptiny}
\end{itemize}
\end{frame}
\gr
%%%%%%%%%%%%%%%%%%%%%%%%%%%%%%%%%%%%%%%%%%%%%%%%%%%%%%%%%%%%%%%%%%%%%%%%%%%%%%%
%%%%%%%%%%%%%%%%%%%%%%%%%%%%%%%%%%%%%%%%%%%%%%%%%%%%%%%%%%%%%%%%%%%%%%%%%%%%%%%
%%%%%%%%%%%%%%%%%%%%%%%%%%%%%%%%%%%%%%%%%%%%%%%%%%%%%%%%%%%%%%%%%%%%%%%%%%%%%%%
\begin{frame}[fragile]{Ενδιάμεσα: Πακέτα}

\begin{itemize}
\item Όλες οι εντολές και τα περιβάλλοντα που χρησιμοποιήσαμε μέχρι στιγμής είναι ενσωματωμένα στο\LaTeX.

\item Τα \emph{Πακέτα} είναι βιβλιοθήκες με επιπλέον εντολές και περιβάλλοντα. Υπάρχουν χιλιάδες ελεύθερα διαθέσιμα πακέτα.

\item Πρέπει να φορτώσουμε κάθε ένα από τα πακέτα που θέλουμε να χρησιμοποιήσουμε με μια
εντολή \en \cmdbs{usepackage}\gr  \space στο \en\emph{preamble}\gr .

\item Παράδειγμα: το \en \bftt{amsmath} \gr από την Αμερικανική Μαθηματική Εταιρεία.
\en
\begin{minted}[fontsize=\small,frame=single]{latex}
\documentclass{article}
\usepackage{amsmath} % preamble
\begin{document}
% now we can use commands from amsmath here...
\end{document}
\end{minted}
\end{itemize}
\end{frame}
\gr
%%%%%%%%%%%%%%%%%%%%%%%%%%%%%%%%%%%%%%%%%%%%%%%%%%%%%%%%%%%%%%%%%%%%%%%%%%%%%%%
%%%%%%%%%%%%%%%%%%%%%%%%%%%%%%%%%%%%%%%%%%%%%%%%%%%%%%%%%%%%%%%%%%%%%%%%%%%%%%%
%%%%%%%%%%%%%%%%%%%%%%%%%%%%%%%%%%%%%%%%%%%%%%%%%%%%%%%%%%%%%%%%%%%%%%%%%%%%%%%
\begin{frame}[fragile]{\insertsubsection{}: Παραδείγματα  \en \bftt{amsmath}\gr}
\begin{itemize}
\item Χρησιμοποιήστε \en \bftt{equation*}\gr (``εξίσωση-αστέρι'') για μη αριθμημένες εξισώσεις.
\en
\begin{exampletwouptiny}
\begin{equation*}
  \Omega = \sum_{k=1}^{n} \omega_k
\end{equation*}
\end{exampletwouptiny}
\gr
\item Το \LaTeX{} αντιμετωπίζει τα διπλανά γράμματα ως μεταβλητές πολλαπλασιασμένες μεταξύ τους, οι οποίες
δεν είναι πάντα αυτό που θέλεις. Το \bftt{amsmath} ορίζει εντολές για πολλούς κοινούς
μαθηματικούς τελεστές.
\en
\begin{exampletwouptiny}
\begin{equation*} % bad!
 min_{x,y} (1-x)^2 + 100(y-x^2)^2
\end{equation*}
\begin{equation*} % good!
\min_{x,y}{(1-x)^2 + 100(y-x^2)^2}
\end{equation*}
\end{exampletwouptiny}
\gr
\item Μπορείτε να χρησιμοποιήσετε επίσης το \en \cmdbs{operatorname}\gr 
\en
\begin{exampletwouptiny}
\begin{equation*}
\beta_i =
\frac{\operatorname{Cov}(R_i, R_m)}
     {\operatorname{Var}(R_m)}
\end{equation*}
\end{exampletwouptiny}
\end{itemize}
\end{frame}
\gr
%%%%%%%%%%%%%%%%%%%%%%%%%%%%%%%%%%%%%%%%%%%%%%%%%%%%%%%%%%%%%%%%%%%%%%%%%%%%%%%
%%%%%%%%%%%%%%%%%%%%%%%%%%%%%%%%%%%%%%%%%%%%%%%%%%%%%%%%%%%%%%%%%%%%%%%%%%%%%%%
%%%%%%%%%%%%%%%%%%%%%%%%%%%%%%%%%%%%%%%%%%%%%%%%%%%%%%%%%%%%%%%%%%%%%%%%%%%%%%%
\begin{frame}[fragile]{\insertsubsection{}: Παραδείγματα  \en \bftt{amsmath}\gr}
\begin{itemize}{\small
\item Ευθυγραμμίστε μια ακολουθία εξισώσεων στο σύμβολο του ίσου
\begin{align*}
(x+1)^3 &= (x+1)(x+1)(x+1) \\
        &= (x+1)(x^2 + 2x + 1) \\
        &= x^3 + 3x^2 + 3x + 1
\end{align*}
με το περιβάλλον \en \bftt{align*}\gr.

% for whatever reason, this doesn't play well with the twoup environment
\en
\begin{minted}[fontsize=\small,frame=single]{latex}
\begin{align*}
(x+1)^3 &= (x+1)(x+1)(x+1) \\
        &= (x+1)(x^2 + 2x + 1) \\
        &= x^3 + 3x^2 + 3x + 1
\end{align*}
\end{minted}
\gr
\item Ένα συμπλεκτικό σύμβολο \keystrokebftt{\&} χωρίζει την αριστερή στήλη (πριν από το
$=$) από τη δεξιά στήλη (μετά το $=$).
\item Μια διπλή ανάστροφη κάθετο \en \keystrokebftt{\bs}\keystrokebftt{\bs}\gr ξεκινά μια νέα
γραμμή.
}\end{itemize}
\end{frame}


%%%%%%%%%%%%%%%%%%%%%%%%%%%%%%%%%%%%%%%%%%%%%%%%%%%%%%%%%%%%%%%%%%%%%%%%%%%%%%%
%%%%%%%%%%%%%%%%%%%%%%%%%%%%%%%%%%%%%%%%%%%%%%%%%%%%%%%%%%%%%%%%%%%%%%%%%%%%%%%
%%%%%%%%%%%%%%%%%%%%%%%%%%%%%%%%%%%%%%%%%%%%%%%%%%%%%%%%%%%%%%%%%%%%%%%%%%%%%%%
\begin{frame}[fragile]{Άσκηση στοιχειοθέτησης 2}

\begin{block}{Στοιχειοθέτηση στο \LaTeX:}
Έστω $X_1, X_2, \ldots, X_n$ να είναι μια ακολουθία ανεξάρτητων και πανομοιότυπα
κατανεμημένες τυχαίες μεταβλητές με $\operatorname{E}[X_i] = \mu$ και
$\operatorname{Var}[X_i] = \sigma^2 < \infty$, και έστω
\begin{equation*}
S_n = \frac{1}{n}\sum_{i=1}^{n} X_i
\end{equation*}
δηλώνουν τη μέση τιμή τους. Στη συνέχεια, καθώς το $n$ πλησιάζει το άπειρο, οι τυχαίες μεταβλητές
$\sqrt{n}(S_n - \mu)$ συγκλίνουν στην κατανομή σε ένα κανονικό $N(0, \sigma^2)$.
\end{block}
\vskip 2ex
\begin{center}
\fbox{\href{\wlnewdoc{basics-exercise-2.tex}}{%
Κάντε κλικ για να ανοίξετε αυτήν την άσκηση στο \en \wllogo{}}}\gr
\end{center}
\begin{itemize}
\item Υπόδειξη: η εντολή για το $\infty$ είναι \en \cmdbs{infty}\gr.
\item Μόλις δοκιμάσετε,
\fbox{\href{\wlnewdoc{basics-exercise-2-solution.tex}}{%
κάντε κλικ εδώ για να δείτε τη λύση μου}}.
\end{itemize}
\end{frame}

%%%%%%%%%%%%%%%%%%%%%%%%%%%%%%%%%%%%%%%%%%%%%%%%%%%%%%%%%%%%%%%%%%%%%%%%%%%%%%%
%%%%%%%%%%%%%%%%%%%%%%%%%%%%%%%%%%%%%%%%%%%%%%%%%%%%%%%%%%%%%%%%%%%%%%%%%%%%%%%
%%%%%%%%%%%%%%%%%%%%%%%%%%%%%%%%%%%%%%%%%%%%%%%%%%%%%%%%%%%%%%%%%%%%%%%%%%%%%%%
\begin{frame}{Τέλος Μέρους 1}
\begin{itemize}
\item Συγχαρητήρια! Έχετε ήδη μάθει πώς να \ldots
\begin{itemize}
\item Πληκτρολογείτε κείμενο στο \LaTeX.
\item Χρησιμοποιείτε πολλές διαφορετικές εντολές.
\item Χειρίζεστε τα λάθη όταν προκύπτουν.
\item Πληκτρολογείτε όμορφα μαθηματικά κείμενα.
\item Χρησιμοποιείτεπολλά διαφορετικά περιβάλλοντα.
\item Φορτώνετε πακέτα.
\end{itemize}
\item Αυτό είναι υπέροχο!
\item Στο Μέρος 2, θα δούμε πώς να χρησιμοποιήσετε το \LaTeX{} για τη σύνταξη δομημένων εγγράφων
με ενότητες, παραπομπές, σχήματα, πίνακες και βιβλιογραφίες.\\



\end{itemize}

\end{frame}

\end{document}
