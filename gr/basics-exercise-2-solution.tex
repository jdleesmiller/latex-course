\documentclass{article}
\usepackage{amsmath}
\usepackage[utf8]{inputenc}
\usepackage[greek,english]{babel}

\newcommand{\en}{\selectlanguage{english}}
\newcommand{\gr}{\selectlanguage{greek}}
\begin{document}
\gr
Έστω $X_1, X_2, \ldots, X_n$ να είναι μια ακολουθία ανεξάρτητων και
πανομοιότυπα κατανεμημένες τυχαίες μεταβλητές με
$\operatorname{E}[X_i] = \mu$ με
$\operatorname{Var}[X_i] = \sigma^2 < \infty$, και έστω
\begin{equation*}
S_n = \frac{1}{n}\sum_{i=1}^{n} X_i
\end{equation*}
δηλώνουν τη μέση τιμή τους. Στη συνέχεια, καθώς το $n$ πλησιάζει το άπειρο, οι τυχαίες μεταβλητές $\sqrt{n}(S_n - \mu)$ cσυγκλίνουν στην κατανομή σε ένα κανονικό $N(0, \sigma^2)$.

% bonus points: the N for normal is usually set in a caligraphic
% font; you can get this using $\mathcal{N}(0, \sigma^2)$.

\end{document}
