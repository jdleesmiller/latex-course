%!TEX program = xelatex
\documentclass{beamer}

%
% Common preamble for all three parts.
%

\usepackage[english]{babel}
\usepackage{amsmath}
\usepackage{color}
\usepackage{minted}
\usepackage{hyperref}
\usepackage{multicol}
\usepackage{tabularx}
\usepackage{tikz}

% only inline todonotes work
\usepackage{xkeyval}
\usepackage[textsize=small]{todonotes}
\presetkeys{todonotes}{inline}{}

\usetikzlibrary{shapes,arrows,positioning,shadows}

% no nav buttons
\usenavigationsymbolstemplate{}

\newcommand{\bftt}[1]{\textbf{\texttt{#1}}}
\newcommand{\comment}[1]{{\color[HTML]{008080}\textit{\textbf{\texttt{#1}}}}}
\newcommand{\cmd}[1]{{\color[HTML]{008000}\bftt{#1}}}
\newcommand{\bs}{\char`\\}
\newcommand{\cmdbs}[1]{\cmd{\bs#1}}
\newcommand{\lcb}{\char '173}
\newcommand{\rcb}{\char '175}
\newcommand{\cmdbegin}[1]{\cmdbs{begin\lcb}\bftt{#1}\cmd{\rcb}}
\newcommand{\cmdend}[1]{\cmdbs{end\lcb}\bftt{#1}\cmd{\rcb}}

\newcommand{\wllogo}{\textbf{write\textrm{\LaTeX}}}

% this is where the example source files are loaded from
% do not include a trailing slash
\newcommand{\fileuri}{https://raw.github.com/jdleesmiller/latex-course/master/en}

\newcommand{\wlserver}{https://www.writelatex.com}
\newcommand{\wlnewdoc}[1]{\wlserver/docs?snip\_uri=\fileuri/#1\&splash=none}

\def\tikzname{Ti\emph{k}Z}

% from http://tex.stackexchange.com/questions/5226/keyboard-font-for-latex
\newcommand*\keystroke[1]{%
  \tikz[baseline=(key.base)]
    \node[%
      draw,
      fill=white,
      drop shadow={shadow xshift=0.25ex,shadow yshift=-0.25ex,fill=black,opacity=0.75},
      rectangle,
      rounded corners=2pt,
      inner sep=1pt,
      line width=0.5pt,
      font=\scriptsize\sffamily
    ](key) {#1\strut}
  ;
}
\newcommand{\keystrokebftt}[1]{\keystroke{\bftt{#1}}}

% stolen from minted.dtx
\newenvironment{exampletwoup}
  {\VerbatimEnvironment
   \begin{VerbatimOut}{example.out}}
  {\end{VerbatimOut}
   \setlength{\parindent}{0pt}
   \fbox{\begin{tabular}{l|l}
   \begin{minipage}{0.55\linewidth}
     \inputminted[fontsize=\small,resetmargins]{latex}{example.out}
   \end{minipage} &
   \begin{minipage}{0.35\linewidth}
     \input{example.out}
   \end{minipage}
   \end{tabular}}}

\newenvironment{exampletwouptiny}
  {\VerbatimEnvironment
   \begin{VerbatimOut}{example.out}}
  {\end{VerbatimOut}
   \setlength{\parindent}{0pt}
   \fbox{\begin{tabular}{l|l}
   \begin{minipage}{0.55\linewidth}
     \inputminted[fontsize=\scriptsize,resetmargins]{latex}{example.out}
   \end{minipage} &
   \begin{minipage}{0.35\linewidth}
     \setlength{\parskip}{6pt plus 1pt minus 1pt}%
     \raggedright\scriptsize\input{example.out}
   \end{minipage}
   \end{tabular}}}

\newenvironment{exampletwouptinynoframe}
  {\VerbatimEnvironment
   \begin{VerbatimOut}{example.out}}
  {\end{VerbatimOut}
   \setlength{\parindent}{0pt}
   \begin{tabular}{l|l}
   \begin{minipage}{0.55\linewidth}
     \inputminted[fontsize=\scriptsize,resetmargins]{latex}{example.out}
   \end{minipage} &
   \begin{minipage}{0.35\linewidth}
     \setlength{\parskip}{6pt plus 1pt minus 1pt}%
     \raggedright\scriptsize\input{example.out}
   \end{minipage}
   \end{tabular}}

\title{An Interactive Introduction to \LaTeX}
\author{Dr John D. Lees-Miller}
\titlegraphic{%
\includegraphics[page=4]{wllogo-series}\\[1em]
\includegraphics[height=24pt]{UoB-logo}
\qquad
\includegraphics[height=24pt]{setsquared_supported}
}



\subtitle{Parte 1: O básico}

\begin{document}   

%%%%%%%%%%%%%%%%%%%%%%%%%%%%%%%%%%%%%%%%%%%%%%%%%%%%%%%%%%%%%%%%%%%%%%%%%%%%%%%
%%%%%%%%%%%%%%%%%%%%%%%%%%%%%%%%%%%%%%%%%%%%%%%%%%%%%%%%%%%%%%%%%%%%%%%%%%%%%%%
%%%%%%%%%%%%%%%%%%%%%%%%%%%%%%%%%%%%%%%%%%%%%%%%%%%%%%%%%%%%%%%%%%%%%%%%%%%%%%%
\begin{frame}
\titlepage
\end{frame}

%%%%%%%%%%%%%%%%%%%%%%%%%%%%%%%%%%%%%%%%%%%%%%%%%%%%%%%%%%%%%%%%%%%%%%%%%%%%%%%
%%%%%%%%%%%%%%%%%%%%%%%%%%%%%%%%%%%%%%%%%%%%%%%%%%%%%%%%%%%%%%%%%%%%%%%%%%%%%%%
%%%%%%%%%%%%%%%%%%%%%%%%%%%%%%%%%%%%%%%%%%%%%%%%%%%%%%%%%%%%%%%%%%%%%%%%%%%%%%%
\begin{frame}{Por que utilizar \LaTeX}

\begin{itemize}
  \item Produz  textos esteticamente agradáveis e bem formatados
  \begin{itemize}
  \item Especialmente textos matemáticos
\end{itemize}
  \item Foi criado por cientistas, para cientistas
  \begin{itemize}
  \item Uma grande comunidade que atualiza e faz novos pacotes
\end{itemize}
  \item É bem poderoso --- pode ser estendido
  \begin{itemize}
    \item Pacotes para artigos, apresentações, planilhas, \ldots
  \end{itemize}
\item Multi plataforma (MSW, Mac OS, Unix)
\item Editores online como o 
\href{https://www.overleaf.com/signup?ref=9868d6ae68e9}{\wllogo}
\end{itemize}



\begin{block}{Dificuldades}
\begin{itemize}
  \item Linguagem de marcação
  \item Mudança de paradigma em relação a como se escreve um texto
  \item Lidar com os erros
  \item É necessário compilar o arquivo .tex para gerar o .pdf.
\end{itemize}
\end{block}



\end{frame}

%%%%%%%%%%%%%%%%%%%%%%%%%%%%%%%%%%%%%%%%%%%%%%%%%%%%%%%%%%%%%%%%%%%%%%%%%%%%%%%
%%%%%%%%%%%%%%%%%%%%%%%%%%%%%%%%%%%%%%%%%%%%%%%%%%%%%%%%%%%%%%%%%%%%%%%%%%%%%%%
%%%%%%%%%%%%%%%%%%%%%%%%%%%%%%%%%%%%%%%%%%%%%%%%%%%%%%%%%%%%%%%%%%%%%%%%%%%%%%%
\begin{frame}[fragile]{Como funciona?}
\begin{itemize}
  \item Você escreve o documento em \texttt{texto puro} com
 \cmd{comandos} que descrevem sua estrutura ou 
  significado
  \item O programa \texttt{latex} processa seu texto e os comandos para produzir
  um documento esteticamente bem formatado.

\end{itemize}
\vskip 2ex
\begin{center}
\begin{minted}[frame=single]{latex}
A chuva na Amazônia \emph{cai} na horizontal.
\end{minted}
\vskip 2ex
\tikz\node[single arrow,fill=gray,font=\ttfamily\bfseries,%
  rotate=270,xshift=-1em]{latex};
\vskip 2ex
\fbox{A chuva na Amazônia \emph{cai} na horizontal.}
\end{center}
\end{frame} 


%%%%%%%%%%%%%%%%%%%%%%%%%%%%%%%%%%%%%%%%%%%%%%%%%%%%%%%%%%%%%%%%%%%%%%%%%%%%%%%
%%%%%%%%%%%%%%%%%%%%%%%%%%%%%%%%%%%%%%%%%%%%%%%%%%%%%%%%%%%%%%%%%%%%%%%%%%%%%%%
%%%%%%%%%%%%%%%%%%%%%%%%%%%%%%%%%%%%%%%%%%%%%%%%%%%%%%%%%%%%%%%%%%%%%%%%%%%%%%%
\begin{frame}[fragile]{Mais exemplos e seus resultados\ldots}
\begin{exampletwoup}
\begin{itemize}
\item Café
\item Leite
\item Bolacha
\end{itemize}
\end{exampletwoup}
\vskip 2ex
\begin{exampletwoup}
\begin{figure}
\includegraphics{gerbil}
\end{figure}
\end{exampletwoup}
\vskip 2ex
\begin{exampletwoup}
\begin{equation}
\alpha + \beta + 1
\end{equation}
\end{exampletwoup}

\tiny{Licença para imagem: \href{https://pixabay.com/en/animal-apple-attractive-beautiful-1239390/}{CC0}}
\end{frame}


%%%%%%%%%%%%%%%%%%%%%%%%%%%%%%%%%%%%%%%%%%%%%%%%%%%%%%%%%%%%%%%%%%%%%%%%%%%%%%%
%%%%%%%%%%%%%%%%%%%%%%%%%%%%%%%%%%%%%%%%%%%%%%%%%%%%%%%%%%%%%%%%%%%%%%%%%%%%%%%
%%%%%%%%%%%%%%%%%%%%%%%%%%%%%%%%%%%%%%%%%%%%%%%%%%%%%%%%%%%%%%%%%%%%%%%%%%%%%%%
\begin{frame}[fragile]{Mudança de atitude}

\begin{itemize}
\item Utilize comandos para descrever `o que é', e não  `como deve parecer'.
\item Concentre-se no conteúdo.
\item Deixe que o \LaTeX{} faça seu trabalho.
\end{itemize}
\end{frame}

%%%%%%%%%%%%%%%%%%%%%%%%%%%%%%%%%%%%%%%%%%%%%%%%%%%%%%%%%%%%%%%%%%%%%%%%%%%%%%%
%%%%%%%%%%%%%%%%%%%%%%%%%%%%%%%%%%%%%%%%%%%%%%%%%%%%%%%%%%%%%%%%%%%%%%%%%%%%%%%
%%%%%%%%%%%%%%%%%%%%%%%%%%%%%%%%%%%%%%%%%%%%%%%%%%%%%%%%%%%%%%%%%%%%%%%%%%%%%%%
\section{O Básico}
%%%%%%%%%%%%%%%%%%%%%%%%%%%%%%%%%%%%%%%%%%%%%%%%%%%%%%%%%%%%%%%%%%%%%%%%%%%%%%%
%%%%%%%%%%%%%%%%%%%%%%%%%%%%%%%%%%%%%%%%%%%%%%%%%%%%%%%%%%%%%%%%%%%%%%%%%%%%%%%
%%%%%%%%%%%%%%%%%%%%%%%%%%%%%%%%%%%%%%%%%%%%%%%%%%%%%%%%%%%%%%%%%%%%%%%%%%%%%%%
\subsection{Pra começo de conversa}
\begin{frame}[fragile]{\insertsubsection}
\begin{itemize}
\item Um documento  \LaTeX{} básico e mínimo em português do Brasil:
\inputminted[frame=single]{latex}{basics.tex}
\item Todo comando começa com uma \emph{barra invertida} \keystrokebftt{\bs}.
\item Todo documento começa com o comando
\cmdbs{documentclass} .
\item o \emph{argumento} que fica entre chaves \keystrokebftt{\{}
\keystrokebftt{\}} diz ao \LaTeX{} que tipo de documento você está criando:
an \bftt{article}.
\item O símbolo de porcentagem \keystrokebftt{\%} inicia um  \emph{comentário}
--- \LaTeX{} irá ignorar tudo que vem após o comentário.
\end{itemize}
\end{frame}

%%%%%%%%%%%%%%%%%%%%%%%%%%%%%%%%%%%%%%%%%%%%%%%%%%%%%%%%%%%%%%%%%%%%%%%%%%%%%%%
%%%%%%%%%%%%%%%%%%%%%%%%%%%%%%%%%%%%%%%%%%%%%%%%%%%%%%%%%%%%%%%%%%%%%%%%%%%%%%%
%%%%%%%%%%%%%%%%%%%%%%%%%%%%%%%%%%%%%%%%%%%%%%%%%%%%%%%%%%%%%%%%%%%%%%%%%%%%%%%
\begin{frame}[fragile]{\insertsubsection{} com \wllogo}
\begin{itemize}
\item \wllogo{} é um  \emph{site} para escrever documentos em 
\LaTeX{}.
\item O \emph{site} `compila' seu \LaTeX{} automaticamente e mostra os
resultados, isto é, o próprio PDF.
\vskip 2em
\begin{center}
\fbox{\href{\wlnewdoc{basics.tex}}{%
Clique aqui para abrir um documento  no \wllogo{}}}
\\[.5ex]
{\small Para melhores resultados,  utilize
\href{http://www.google.com/chrome}{Google Chrome} ou uma versão recente do 
\href{http://www.mozilla.org/pt-BR/firefox/new/}{FireFox}. }
\end{center}
\vskip 2ex
\item Conforme avancemos para os próximos \emph{slides}, tente os exemplos,
digitando-os no documento exemplo no \wllogo{}.
\item \textbf{Para aprender você deve tentar durante o curso!}
\end{itemize}
\end{frame}


%%%%%%%%%%%%%%%%%%%%%%%%%%%%%%%%%%%%%%%%%%%%%%%%%%%%%%%%%%%%%%%%%%%%%%%%%%%%%%%
%%%%%%%%%%%%%%%%%%%%%%%%%%%%%%%%%%%%%%%%%%%%%%%%%%%%%%%%%%%%%%%%%%%%%%%%%%%%%%%
%%%%%%%%%%%%%%%%%%%%%%%%%%%%%%%%%%%%%%%%%%%%%%%%%%%%%%%%%%%%%%%%%%%%%%%%%%%%%%%
\subsection{Digitando Texto}
\begin{frame}[fragile]{\insertsubsection{}}
\small
\begin{itemize}
\item Digite seu texto entre os comandos \cmdbegin{document} e
\cmdend{document}.
\item Na maior parte do tempo, você pode digitar o texto normalmente.
\begin{exampletwouptiny}
Palavras são separadas por um ou mais
espaços.

Parágrafos são separados por uma
ou mais linhas em branco.
\end{exampletwouptiny}
\item Espaços a mais no código-fonte serão eliminados no arquivo-saída.
\begin{exampletwouptiny}
A    chuva   na       Amazônia
cai    na Horizontal.
\end{exampletwouptiny}
\end{itemize}
\end{frame}


%%%%%%%%%%%%%%%%%%%%%%%%%%%%%%%%%%%%%%%%%%%%%%%%%%%%%%%%%%%%%%%%%%%%%%%%%%%%%%%
%%%%%%%%%%%%%%%%%%%%%%%%%%%%%%%%%%%%%%%%%%%%%%%%%%%%%%%%%%%%%%%%%%%%%%%%%%%%%%%
%%%%%%%%%%%%%%%%%%%%%%%%%%%%%%%%%%%%%%%%%%%%%%%%%%%%%%%%%%%%%%%%%%%%%%%%%%%%%%%
\begin{frame}[fragile]{\insertsubsection{}: Advertências}
\small
\begin{itemize}
\item Marcas de citação são um pouco chatas:\\ use uma
crase \keystroke{`} na esquerda e um apóstrofo 
\keystroke{'} na direita.
\begin{exampletwouptiny}
Aspas simples: `texto'.

Aspas duplas: ``texto''.
\end{exampletwouptiny}

\item Alguns caracteres comuns tem significado especial no  \LaTeX:\\[1ex]
\begin{tabular}{ll}
\keystrokebftt{\%} & porcentagem              \\
\keystrokebftt{\#} & \emph{hashtag} (jogo da velha) \\
\keystrokebftt{\&} & ``E'' comercial                 \\
\keystrokebftt{\$} & cifrão               \\
\keystrokebftt{\{} e \keystrokebftt{\}}  & abre e fecha chaves               \\
\end{tabular}
\item Se você apenas digitá-los, você terá um erro. Se você quiser que algum
deles apareça, terá que utilizar uma barra invertida antes do símbolo.
\begin{exampletwoup}
\$\%\&\#\{\} 
\end{exampletwoup}
\end{itemize}
\end{frame}

%%%%%%%%%%%%%%%%%%%%%%%%%%%%%%%%%%%%%%%%%%%%%%%%%%%%%%%%%%%%%%%%%%%%%%%%%%%%%%%
%%%%%%%%%%%%%%%%%%%%%%%%%%%%%%%%%%%%%%%%%%%%%%%%%%%%%%%%%%%%%%%%%%%%%%%%%%%%%%%
%%%%%%%%%%%%%%%%%%%%%%%%%%%%%%%%%%%%%%%%%%%%%%%%%%%%%%%%%%%%%%%%%%%%%%%%%%%%%%%
\begin{frame}[fragile]{Lidando com Erros}
\begin{itemize}
\item \LaTeX{} pode se confundir ao compilar seu documento. Se isso acontecer,
o programa parará e indicará um erro, o qual deverá ser corrigido antes de produzir o documento.
\item Por exemplo, se você digitar o comando \cmdbs{emph} como \cmdbs{meph},
\LaTeX{} irá parar com um erro ``\emph{undefined control sequence}'' (sequência de
controle desconhecida) , já que ``meph'' não é um comando conhecido.
\end{itemize}
\begin{block}{Recomendações para lidar com Erros}
\begin{enumerate}
\item Não entre em pânico. Erros  acontecem e são comuns.
\item Corrija-os assim que eles aparecerem: --- se o que você digitou causou
erro, deve \emph{debugar} a partir daquela linha.
\item Se houver muitos erros, comece pelo primeiro:  --- a causa de um erro
pode ser um outro erro anterior.
\end{enumerate}
\end{block}
\end{frame}

%%%%%%%%%%%%%%%%%%%%%%%%%%%%%%%%%%%%%%%%%%%%%%%%%%%%%%%%%%%%%%%%%%%%%%%%%%%%%%%
%%%%%%%%%%%%%%%%%%%%%%%%%%%%%%%%%%%%%%%%%%%%%%%%%%%%%%%%%%%%%%%%%%%%%%%%%%%%%%%
%%%%%%%%%%%%%%%%%%%%%%%%%%%%%%%%%%%%%%%%%%%%%%%%%%%%%%%%%%%%%%%%%%%%%%%%%%%%%%%
\begin{frame}[fragile]{Exercício de digitação 1}

\begin{block}{Escreva o texto abaixo em \LaTeX:
\footnote{\url{http://pt.wikipedia.org/wiki/Brasil}}}
Brasil, oficialmente ``República Federativa do Brasil''  é o maior país da
América do Sul e da região da América latina. [...] O setor de serviços responde pela
maior parte do PIB, com 66,8\%, seguido pelo setor industrial, com 29,7\%
(estimativa para 2007), enquanto a agricultura representa 3,5\% (2008 est). A
força de trabalho brasileira é estimada em R\$ 100,77 milhões, dos quais 10\%
são ocupados na agricultura, 19\% no setor da indústria e 71\% no setor de
serviços.
\end{block}
\vskip 2ex
\begin{center}
\fbox{\href{\wlnewdoc{basics-exercise-1.tex}}{%
Clique aqui para abrir os exercícios no \wllogo{}}}
\end{center}

\begin{itemize}
\item Dica: perceba os caracteres que tem significado especial!
\item Uma vez que tenha  terminado,
\fbox{\href{\wlnewdoc{basics-exercise-1-solution.tex}}{%
veja aqui minha solução}}.
\end{itemize}
\end{frame}

%%%%%%%%%%%%%%%%%%%%%%%%%%%%%%%%%%%%%%%%%%%%%%%%%%%%%%%%%%%%%%%%%%%%%%%%%%%%%%%
%%%%%%%%%%%%%%%%%%%%%%%%%%%%%%%%%%%%%%%%%%%%%%%%%%%%%%%%%%%%%%%%%%%%%%%%%%%%%%%
%%%%%%%%%%%%%%%%%%%%%%%%%%%%%%%%%%%%%%%%%%%%%%%%%%%%%%%%%%%%%%%%%%%%%%%%%%%%%%%
\subsection{Digitando Matemática}
\begin{frame}[fragile]{\insertsubsection{}: Cifrão (\$)}
\begin{itemize}
\item Por que o cifrão  \keystrokebftt{\$} é um caractere especial? Porque
o utilizamos para marcar textos matemáticos.\\[1ex]
\begin{exampletwouptiny}
% não tão bom:
Seja a e b inteiros positivos 
distintos, e seja  
c = a - b + 1.

% muito melhor:
Seja $a$ e $b$ inteiros positivos 
distintos, e seja
$c = a - b + 1$.
\end{exampletwouptiny}
\item Sempre utilize cifrão em pares --- um para o começo do texto matemático e
outro para o final.
\item \LaTeX{} maneja o espaçamento automaticamente; ele ignora os seus espaços.
\begin{exampletwouptiny}
Seja $y=mx+b$  \ldots

Seja $y = m x + b$  \ldots
\end{exampletwouptiny}
\end{itemize}
\end{frame}

%%%%%%%%%%%%%%%%%%%%%%%%%%%%%%%%%%%%%%%%%%%%%%%%%%%%%%%%%%%%%%%%%%%%%%%%%%%%%%%
%%%%%%%%%%%%%%%%%%%%%%%%%%%%%%%%%%%%%%%%%%%%%%%%%%%%%%%%%%%%%%%%%%%%%%%%%%%%%%%
%%%%%%%%%%%%%%%%%%%%%%%%%%%%%%%%%%%%%%%%%%%%%%%%%%%%%%%%%%%%%%%%%%%%%%%%%%%%%%%
\begin{frame}[fragile]{\insertsubsection{}: Notação}
\begin{itemize}
\item Use circunflexo \keystrokebftt{\textasciicircum} para sobrescritos e
\emph{underline} \keystrokebftt{\_} para índices.
\begin{exampletwouptiny}
$y = c_2 x^2 + c_1 x + c_0$
\end{exampletwouptiny}
\vskip 2ex

\item Use chaves \keystrokebftt{\{} \keystrokebftt{\}} para agrupar
sobrescritos e índices.
\begin{exampletwouptiny}
$F_n = F_n-1 + F_n-2$     % oops!

$F_n = F_{n-1} + F_{n-2}$ % ok!
\end{exampletwouptiny}
\vskip 2ex

\item Há também comandos para letras Gregas e para notações comuns.
\begin{exampletwouptiny}
$\mu = A e^{Q/RT}$

$\Omega = \sum_{k=1}^{n} \omega_k$
\end{exampletwouptiny}
\end{itemize}
\end{frame}

%%%%%%%%%%%%%%%%%%%%%%%%%%%%%%%%%%%%%%%%%%%%%%%%%%%%%%%%%%%%%%%%%%%%%%%%%%%%%%%
%%%%%%%%%%%%%%%%%%%%%%%%%%%%%%%%%%%%%%%%%%%%%%%%%%%%%%%%%%%%%%%%%%%%%%%%%%%%%%%
%%%%%%%%%%%%%%%%%%%%%%%%%%%%%%%%%%%%%%%%%%%%%%%%%%%%%%%%%%%%%%%%%%%%%%%%%%%%%%%
\begin{frame}[fragile]{\insertsubsection{}: Equações centralizadas}
\begin{itemize}
\item Se algo for grande e amedrontador, \emph{exiba-o} em sua própria linha
utilizando \cmdbegin{equation} e \cmdend{equation}.\\[2ex]
\begin{exampletwouptiny}
As raízes de uma equação quadrática
são dadas por
\begin{equation}
x = \frac{-b \pm \sqrt{b^2 - 4ac}}
         {2a}
\end{equation}
em que $a$, $b$ e $c$ são \ldots
\end{exampletwouptiny}
\vskip 1em
{\scriptsize Alerta: \LaTeX{} na maioria das vezes ignora espaços no ambiente
matemático, porém não pode manejar linhas em branco em equações --- não pule
linhas dentro de ambientes matemáticos.}
\end{itemize}
\end{frame}

%%%%%%%%%%%%%%%%%%%%%%%%%%%%%%%%%%%%%%%%%%%%%%%%%%%%%%%%%%%%%%%%%%%%%%%%%%%%%%%
%%%%%%%%%%%%%%%%%%%%%%%%%%%%%%%%%%%%%%%%%%%%%%%%%%%%%%%%%%%%%%%%%%%%%%%%%%%%%%%
%%%%%%%%%%%%%%%%%%%%%%%%%%%%%%%%%%%%%%%%%%%%%%%%%%%%%%%%%%%%%%%%%%%%%%%%%%%%%%%
\begin{frame}[fragile]{Interlúdio: Ambientes}
\begin{itemize}
\item \bftt{equation} é um \emph{ambiente} --- um contexto.
\item Um comando pode produzir diferentes resultados em contextos diferentes.
\begin{exampletwouptiny}
Podemos escrever
$ \Omega = \sum_{k=1}^{n} \omega_k $
no texto, ou podemos escrever
\begin{equation}
  \Omega = \sum_{k=1}^{n} \omega_k
\end{equation}
para exibi-lo.
\end{exampletwouptiny}
\vskip 2ex
\item Note que  $\Sigma$ é maior no ambiente \bftt{equation}, 
e como sobrescritos e índices mudam de posição, embora você tenha utilizado
os mesmos comandos

\end{itemize}
\end{frame}

%%%%%%%%%%%%%%%%%%%%%%%%%%%%%%%%%%%%%%%%%%%%%%%%%%%%%%%%%%%%%%%%%%%%%%%%%%%%%%%
%%%%%%%%%%%%%%%%%%%%%%%%%%%%%%%%%%%%%%%%%%%%%%%%%%%%%%%%%%%%%%%%%%%%%%%%%%%%%%%
%%%%%%%%%%%%%%%%%%%%%%%%%%%%%%%%%%%%%%%%%%%%%%%%%%%%%%%%%%%%%%%%%%%%%%%%%%%%%%%
\begin{frame}[fragile]{Interlúdio: Ambientes}
\begin{itemize}
\item Os comandos \cmdbs{begin} e \cmdbs{end} são utilizados para criar
diferentes ambientes.
\vskip 2ex

\item Os ambientes \bftt{itemize} e \bftt{enumerate} geram listas.
\begin{exampletwouptiny}
\begin{itemize} % para itens
\item Bolachas
\item Cafés
\end{itemize}

\begin{enumerate} % para números
\item Bolachas
\item Cafés
\end{enumerate}
\end{exampletwouptiny}
\end{itemize}
\end{frame}

%%%%%%%%%%%%%%%%%%%%%%%%%%%%%%%%%%%%%%%%%%%%%%%%%%%%%%%%%%%%%%%%%%%%%%%%%%%%%%%
%%%%%%%%%%%%%%%%%%%%%%%%%%%%%%%%%%%%%%%%%%%%%%%%%%%%%%%%%%%%%%%%%%%%%%%%%%%%%%%
%%%%%%%%%%%%%%%%%%%%%%%%%%%%%%%%%%%%%%%%%%%%%%%%%%%%%%%%%%%%%%%%%%%%%%%%%%%%%%%
\begin{frame}[fragile]{Interlúdio: Pacotes}

\begin{itemize}
\item Todos os comandos que utilizamos estão contidos nas distribuições de
\LaTeX.

\item \emph{Pacotes} são bibliotecas com comandos e ambientes extra. Há milhares
de pacotes disponíveis.

\item Devemos chamar cada pacote que queremos utilizar com um comando
\cmdbs{usepackage} no \emph{preâmbulo}.

\item Exemplo: \bftt{amsmath} da  American Mathematical Society.
\begin{minted}[fontsize=\small,frame=single]{latex}
\documentclass{article}
\usepackage{amsmath} % preâmbulo
\begin{document}
% já podemos usar os comandos do pacote amsmath aqui
\end{document}
\end{minted}
\end{itemize}
\end{frame}
%%%%%%%%%%%%%%%%%%%%%%%%%%%%%%%%%%%%%%%%%%%%%%%%%%%%%%%%%%%%%%%%%%%%%%%%%%%%%%%
%%%%%%%%%%%%%%%%%%%%%%%%%%%%%%%%%%%%%%%%%%%%%%%%%%%%%%%%%%%%%%%%%%%%%%%%%%%%%%%
%%%%%%%%%%%%%%%%%%%%%%%%%%%%%%%%%%%%%%%%%%%%%%%%%%%%%%%%%%%%%%%%%%%%%%%%%%%%%%%
\begin{frame}[fragile]{\insertsubsection{}: Exemplos com \bftt{amsmath}}
\begin{itemize}
\item Use \bftt{equation*} (``equation-asterisco'') para equações não numeradas.
\begin{exampletwouptiny}
\begin{equation*}
  \Omega = \sum_{k=1}^{n} \omega_k
\end{equation*}
\end{exampletwouptiny}
\item \LaTeX{} trata letras adjacentes como variáveis múltiplas multiplicadas, o
que nem sempre é o que você deseja. 
\bftt{amsmath} define vários operadores matemáticos comuns.
\begin{exampletwouptiny}
\begin{equation*} % ruim!
 min_{x,y} (1-x)^2 + 100(y-x^2)^2 
\end{equation*}
\begin{equation*} % bom!
\min_{x,y}{(1-x)^2 + 100(y-x^2)^2}
\end{equation*}
\end{exampletwouptiny}
\item Pode-se utilizar  \cmdbs{operatorname} pra outros.
\begin{exampletwouptiny}
\begin{equation*}
\beta_i =\operatorname{sen}(\alpha)
\frac{\operatorname{Cov}(R_i, R_m)}
     {\operatorname{Var}(R_m)}
\end{equation*}
\end{exampletwouptiny}
\end{itemize}
\end{frame}

%%%%%%%%%%%%%%%%%%%%%%%%%%%%%%%%%%%%%%%%%%%%%%%%%%%%%%%%%%%%%%%%%%%%%%%%%%%%%%%
%%%%%%%%%%%%%%%%%%%%%%%%%%%%%%%%%%%%%%%%%%%%%%%%%%%%%%%%%%%%%%%%%%%%%%%%%%%%%%%
%%%%%%%%%%%%%%%%%%%%%%%%%%%%%%%%%%%%%%%%%%%%%%%%%%%%%%%%%%%%%%%%%%%%%%%%%%%%%%%
\begin{frame}[fragile]{\insertsubsection{}: Exemplos com \bftt{amsmath}}
\begin{itemize}{\small
\item Alinhe  uma sequência de equações com o símbolo de igualdade
\begin{align*}
(x+1)^3 &= (x+1)(x+1)(x+1) \\
        &= (x+1)(x^2 + 2x + 1) \\
        &= x^3 + 3x^2 + 3x + 1
\end{align*}
com o ambiente \bftt{align*}.

% por alguma razão, isto não fica muito bem
\begin{minted}[fontsize=\small,frame=single]{latex}
\begin{align*}
(x+1)^3 &= (x+1)(x+1)(x+1) \\
        &= (x+1)(x^2 + 2x + 1) \\
        &= x^3 + 3x^2 + 3x + 1
\end{align*}
\end{minted}
\item O e comercial \keystrokebftt{\&} separa a coluna da esquerda (antes do
$=$) da coluna da direta (após o $=$).
\item As barras invertidas duplas
\keystrokebftt{\bs}\keystrokebftt{\bs} iniciam uma nova linha.
}\end{itemize}
\end{frame}

%%%%%%%%%%%%%%%%%%%%%%%%%%%%%%%%%%%%%%%%%%%%%%%%%%%%%%%%%%%%%%%%%%%%%%%%%%%%%%%
%%%%%%%%%%%%%%%%%%%%%%%%%%%%%%%%%%%%%%%%%%%%%%%%%%%%%%%%%%%%%%%%%%%%%%%%%%%%%%%
%%%%%%%%%%%%%%%%%%%%%%%%%%%%%%%%%%%%%%%%%%%%%%%%%%%%%%%%%%%%%%%%%%%%%%%%%%%%%%%
\begin{frame}[fragile]{Exercícios de escrita 2}

\begin{block}{Escreva o texto abaixo em \LaTeX:}
Sejam $X_1, X_2, \ldots, X_n$ uma sequência de variáveis aleatórias
independentes e distribuídas de forma igual com $\operatorname{E}[X_i] = \mu$
e $\operatorname{Var}[X_i] = \sigma^2 < \infty$, e denote
\begin{equation*}
S_n = \frac{1}{n}\sum_{i=1}^{n} X_i
\end{equation*}
suas médias. Então quando $n$ tende ao infinito, as variáveis aleatórias  
$\sqrt{n}(S_n - \mu)$ convergem em distribuição à uma normal $N(0, \sigma^2)$.
\end{block}
\vskip 2ex
\begin{center}
\fbox{\href{\wlnewdoc{basics-exercise-2.tex}}{%
Clique aqui para abrir o exercícios no \wllogo{}}}
\end{center}
\begin{itemize}
\item Dica: use o comando \cmdbs{infty} para $\infty$.
\item Uma vez que tenha tentado,
\fbox{\href{\wlnewdoc{basics-exercise-2-solution.tex}}{%
clique aqui e veja a solução}}.
\end{itemize}
\end{frame}

%%%%%%%%%%%%%%%%%%%%%%%%%%%%%%%%%%%%%%%%%%%%%%%%%%%%%%%%%%%%%%%%%%%%%%%%%%%%%%%
%%%%%%%%%%%%%%%%%%%%%%%%%%%%%%%%%%%%%%%%%%%%%%%%%%%%%%%%%%%%%%%%%%%%%%%%%%%%%%%
%%%%%%%%%%%%%%%%%%%%%%%%%%%%%%%%%%%%%%%%%%%%%%%%%%%%%%%%%%%%%%%%%%%%%%%%%%%%%%%
\begin{frame}{Fim da Parte 1}
\begin{itemize}
\item Muito bem! Você já aprendeu a \ldots
\begin{itemize}
\item Escrever textos em  \LaTeX.
\item Usar vários  comandos.
\item Manejar erros quando eles aparecem.
\item Digitar  textos matemáticos.
\item Usar alguns ambientes diferentes.
\item Chamar pacotes.
\end{itemize}
\item Fantástico!
\item Na Parte 2, veremos como usar  \LaTeX{} para escrever documentos
estruturados com seções, referências cruzadas, figuras, tabelas e referências bibliográficas.
Vejo vocês!
\end{itemize}
\end{frame}

\end{document}
