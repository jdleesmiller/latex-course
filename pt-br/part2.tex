%!TEX program = xelatex
\documentclass{beamer}

%
% Common preamble for all three parts.
%

\usepackage[english]{babel}
\usepackage{amsmath}
\usepackage{color}
\usepackage{minted}
\usepackage{hyperref}
\usepackage{multicol}
\usepackage{tabularx}
\usepackage{tikz}

% only inline todonotes work
\usepackage{xkeyval}
\usepackage[textsize=small]{todonotes}
\presetkeys{todonotes}{inline}{}

\usetikzlibrary{shapes,arrows,positioning,shadows}

% no nav buttons
\usenavigationsymbolstemplate{}

\newcommand{\bftt}[1]{\textbf{\texttt{#1}}}
\newcommand{\comment}[1]{{\color[HTML]{008080}\textit{\textbf{\texttt{#1}}}}}
\newcommand{\cmd}[1]{{\color[HTML]{008000}\bftt{#1}}}
\newcommand{\bs}{\char`\\}
\newcommand{\cmdbs}[1]{\cmd{\bs#1}}
\newcommand{\lcb}{\char '173}
\newcommand{\rcb}{\char '175}
\newcommand{\cmdbegin}[1]{\cmdbs{begin\lcb}\bftt{#1}\cmd{\rcb}}
\newcommand{\cmdend}[1]{\cmdbs{end\lcb}\bftt{#1}\cmd{\rcb}}

\newcommand{\wllogo}{\textbf{write\textrm{\LaTeX}}}

% this is where the example source files are loaded from
% do not include a trailing slash
\newcommand{\fileuri}{https://raw.github.com/jdleesmiller/latex-course/master/en}

\newcommand{\wlserver}{https://www.writelatex.com}
\newcommand{\wlnewdoc}[1]{\wlserver/docs?snip\_uri=\fileuri/#1\&splash=none}

\def\tikzname{Ti\emph{k}Z}

% from http://tex.stackexchange.com/questions/5226/keyboard-font-for-latex
\newcommand*\keystroke[1]{%
  \tikz[baseline=(key.base)]
    \node[%
      draw,
      fill=white,
      drop shadow={shadow xshift=0.25ex,shadow yshift=-0.25ex,fill=black,opacity=0.75},
      rectangle,
      rounded corners=2pt,
      inner sep=1pt,
      line width=0.5pt,
      font=\scriptsize\sffamily
    ](key) {#1\strut}
  ;
}
\newcommand{\keystrokebftt}[1]{\keystroke{\bftt{#1}}}

% stolen from minted.dtx
\newenvironment{exampletwoup}
  {\VerbatimEnvironment
   \begin{VerbatimOut}{example.out}}
  {\end{VerbatimOut}
   \setlength{\parindent}{0pt}
   \fbox{\begin{tabular}{l|l}
   \begin{minipage}{0.55\linewidth}
     \inputminted[fontsize=\small,resetmargins]{latex}{example.out}
   \end{minipage} &
   \begin{minipage}{0.35\linewidth}
     \input{example.out}
   \end{minipage}
   \end{tabular}}}

\newenvironment{exampletwouptiny}
  {\VerbatimEnvironment
   \begin{VerbatimOut}{example.out}}
  {\end{VerbatimOut}
   \setlength{\parindent}{0pt}
   \fbox{\begin{tabular}{l|l}
   \begin{minipage}{0.55\linewidth}
     \inputminted[fontsize=\scriptsize,resetmargins]{latex}{example.out}
   \end{minipage} &
   \begin{minipage}{0.35\linewidth}
     \setlength{\parskip}{6pt plus 1pt minus 1pt}%
     \raggedright\scriptsize\input{example.out}
   \end{minipage}
   \end{tabular}}}

\newenvironment{exampletwouptinynoframe}
  {\VerbatimEnvironment
   \begin{VerbatimOut}{example.out}}
  {\end{VerbatimOut}
   \setlength{\parindent}{0pt}
   \begin{tabular}{l|l}
   \begin{minipage}{0.55\linewidth}
     \inputminted[fontsize=\scriptsize,resetmargins]{latex}{example.out}
   \end{minipage} &
   \begin{minipage}{0.35\linewidth}
     \setlength{\parskip}{6pt plus 1pt minus 1pt}%
     \raggedright\scriptsize\input{example.out}
   \end{minipage}
   \end{tabular}}

\title{An Interactive Introduction to \LaTeX}
\author{Dr John D. Lees-Miller}
\titlegraphic{%
\includegraphics[page=4]{wllogo-series}\\[1em]
\includegraphics[height=24pt]{UoB-logo}
\qquad
\includegraphics[height=24pt]{setsquared_supported}
}



\subtitle{Parte 2: Documentos Estruturados \& muito mais}

\begin{document}

%%%%%%%%%%%%%%%%%%%%%%%%%%%%%%%%%%%%%%%%%%%%%%%%%%%%%%%%%%%%%%%%%%%%%%%%%%%%%%%
%%%%%%%%%%%%%%%%%%%%%%%%%%%%%%%%%%%%%%%%%%%%%%%%%%%%%%%%%%%%%%%%%%%%%%%%%%%%%%%
%%%%%%%%%%%%%%%%%%%%%%%%%%%%%%%%%%%%%%%%%%%%%%%%%%%%%%%%%%%%%%%%%%%%%%%%%%%%%%%
\begin{frame}
\titlepage
\end{frame}

%%%%%%%%%%%%%%%%%%%%%%%%%%%%%%%%%%%%%%%%%%%%%%%%%%%%%%%%%%%%%%%%%%%%%%%%%%%%%%%
%%%%%%%%%%%%%%%%%%%%%%%%%%%%%%%%%%%%%%%%%%%%%%%%%%%%%%%%%%%%%%%%%%%%%%%%%%%%%%%
%%%%%%%%%%%%%%%%%%%%%%%%%%%%%%%%%%%%%%%%%%%%%%%%%%%%%%%%%%%%%%%%%%%%%%%%%%%%%%%
\section{Documentos Estruturados}

%%%%%%%%%%%%%%%%%%%%%%%%%%%%%%%%%%%%%%%%%%%%%%%%%%%%%%%%%%%%%%%%%%%%%%%%%%%%%%%
%%%%%%%%%%%%%%%%%%%%%%%%%%%%%%%%%%%%%%%%%%%%%%%%%%%%%%%%%%%%%%%%%%%%%%%%%%%%%%%
%%%%%%%%%%%%%%%%%%%%%%%%%%%%%%%%%%%%%%%%%%%%%%%%%%%%%%%%%%%%%%%%%%%%%%%%%%%%%%%
\begin{frame}{Sumário}
\begin{multicols}{2}
\tableofcontents[currentsection]
\end{multicols}
\end{frame}

%%%%%%%%%%%%%%%%%%%%%%%%%%%%%%%%%%%%%%%%%%%%%%%%%%%%%%%%%%%%%%%%%%%%%%%%%%%%%%%
%%%%%%%%%%%%%%%%%%%%%%%%%%%%%%%%%%%%%%%%%%%%%%%%%%%%%%%%%%%%%%%%%%%%%%%%%%%%%%%
%%%%%%%%%%%%%%%%%%%%%%%%%%%%%%%%%%%%%%%%%%%%%%%%%%%%%%%%%%%%%%%%%%%%%%%%%%%%%%%
\begin{frame}{\insertsection}
\begin{itemize}
\item Na Parte 1, aprendemos como utilizar comandos e ambientes para digitar textos, matemáticos ou não.
\item Agora, aprenderemos a utilizar comandos e ambientes para estruturar documentos.
\item Você pode tentar os novos comandos no Overleaf:
\end{itemize}
\vskip 2em
\begin{center}
\fbox{\href{\wlnewdoc{basics.tex}}{%
Clique aqui para abrir o documento-exemplo no \wllogo{}}}
\\[1ex]\scriptsize{}
Para melhores resultados, por favor use \href{http://www.google.com/chrome}{Google Chrome} ou \href{http://www.mozilla.org/pt-BR/firefox/new/}{FireFox}.
\end{center}
\vskip 2ex
\begin{itemize}
\item Vamos começar!
\end{itemize}
\end{frame}

%%%%%%%%%%%%%%%%%%%%%%%%%%%%%%%%%%%%%%%%%%%%%%%%%%%%%%%%%%%%%%%%%%%%%%%%%%%%%%%
%%%%%%%%%%%%%%%%%%%%%%%%%%%%%%%%%%%%%%%%%%%%%%%%%%%%%%%%%%%%%%%%%%%%%%%%%%%%%%%
%%%%%%%%%%%%%%%%%%%%%%%%%%%%%%%%%%%%%%%%%%%%%%%%%%%%%%%%%%%%%%%%%%%%%%%%%%%%%%%
\subsection{Título e Resumo}
\begin{frame}[fragile]{\insertsubsection}
\begin{itemize}{\small
\item Informe ao \LaTeX{} no preâmbulo o título, usando o comando \cmdbs{title} e o autor, usando o comando \cmdbs{author}.
\item Então  use \cmdbs{maketitle} no documento para realmente criar o título.
\item Use o ambiente \bftt{abstract} para criar um resumo.
}\end{itemize}
\begin{minipage}{0.45\linewidth}
\inputminted[fontsize=\tiny,frame=single,resetmargins]{latex}%
  {structure-title.tex}
\end{minipage}
\begin{minipage}{0.45\linewidth}
\includegraphics[width=\textwidth,clip,trim=2.2in 7in 2.2in 2in]{structure-title.pdf}
\end{minipage}
\end{frame}

%%%%%%%%%%%%%%%%%%%%%%%%%%%%%%%%%%%%%%%%%%%%%%%%%%%%%%%%%%%%%%%%%%%%%%%%%%%%%%%
%%%%%%%%%%%%%%%%%%%%%%%%%%%%%%%%%%%%%%%%%%%%%%%%%%%%%%%%%%%%%%%%%%%%%%%%%%%%%%%
%%%%%%%%%%%%%%%%%%%%%%%%%%%%%%%%%%%%%%%%%%%%%%%%%%%%%%%%%%%%%%%%%%%%%%%%%%%%%%%
\subsection{Seções}
\begin{frame}{\insertsubsection}
\begin{itemize}{\small
\item Utilize os comandos  \cmdbs{section} e \cmdbs{subsection} para criar seções e subseções.
\item Você pode adivinhar o que os comandos \cmdbs{section*} e \cmdbs{subsection*} fazem?
}\end{itemize}
\begin{minipage}{0.55\linewidth}
\inputminted[fontsize=\tiny,frame=single,resetmargins]{latex}%
  {structure-sections.tex}
\end{minipage}
\begin{minipage}{0.40\linewidth} 
\expandafter\includegraphics\expandafter[width=\textwidth,clip,trim=1.5in 6in 3.5in 1in]{structure-sections.pdf}
\end{minipage}
\end{frame}

%%%%%%%%%%%%%%%%%%%%%%%%%%%%%%%%%%%%%%%%%%%%%%%%%%%%%%%%%%%%%%%%%%%%%%%%%%%%%%%
%%%%%%%%%%%%%%%%%%%%%%%%%%%%%%%%%%%%%%%%%%%%%%%%%%%%%%%%%%%%%%%%%%%%%%%%%%%%%%%
%%%%%%%%%%%%%%%%%%%%%%%%%%%%%%%%%%%%%%%%%%%%%%%%%%%%%%%%%%%%%%%%%%%%%%%%%%%%%%%
\subsection{Rótulos e Referência Cruzada}
\begin{frame}[fragile]{\insertsubsection}
\begin{itemize}{\small
\item Use \cmdbs{label} para rotular e \cmdbs{ref} para referência cruzada automática. 
\item O pacote \bftt{amsmath} disponibiliza o comando \cmdbs{eqref} para referenciar equações.
}\end{itemize}
\begin{minipage}{0.45\linewidth}
\inputminted[fontsize=\tiny,frame=single,resetmargins]{latex}%
  {structure-crossref.tex}
\end{minipage}
\begin{minipage}{0.45\linewidth}
\includegraphics[width=\textwidth,clip,trim=1.8in 6in 1.8in 1in]{structure-crossref.pdf}
\end{minipage}
\end{frame}

%%%%%%%%%%%%%%%%%%%%%%%%%%%%%%%%%%%%%%%%%%%%%%%%%%%%%%%%%%%%%%%%%%%%%%%%%%%%%%%
%%%%%%%%%%%%%%%%%%%%%%%%%%%%%%%%%%%%%%%%%%%%%%%%%%%%%%%%%%%%%%%%%%%%%%%%%%%%%%%
%%%%%%%%%%%%%%%%%%%%%%%%%%%%%%%%%%%%%%%%%%%%%%%%%%%%%%%%%%%%%%%%%%%%%%%%%%%%%%%
\subsection{Exercício}
\begin{frame}[fragile]{Exercícios de um documento estruturado}

\begin{block}{Digite um  artigo curto em \LaTeX:
\footnote{Traduzido a partir de  \url{http://pdos.csail.mit.edu/scigen/}, um gerador de artigos aleatórios.}}
\begin{center}
\fbox{\href{\fileuri/structure-exercise-solution.pdf}{%
Clique aqui para abrir o artigo}}
\end{center}
Faça seu artigo parecer com este aqui. Use os comandos \cmdbs{ref} e \cmdbs{eqref} para evitar escrever explicitamente seção e número de equações no texto.
\end{block}
\vskip 2ex
\begin{center}
\fbox{\href{\wlnewdoc{structure-exercise.tex}}{%
Clique aqui para abrir este exercício no \wllogo{}}}
\end{center}

\begin{itemize}
\item Uma vez que tenha tentado,
\fbox{\href{\wlnewdoc{structure-exercise-solution.tex}}{%
clique aqui para ver a solução}}.
\end{itemize}
\end{frame}

%%%%%%%%%%%%%%%%%%%%%%%%%%%%%%%%%%%%%%%%%%%%%%%%%%%%%%%%%%%%%%%%%%%%%%%%%%%%%%%
%%%%%%%%%%%%%%%%%%%%%%%%%%%%%%%%%%%%%%%%%%%%%%%%%%%%%%%%%%%%%%%%%%%%%%%%%%%%%%%
%%%%%%%%%%%%%%%%%%%%%%%%%%%%%%%%%%%%%%%%%%%%%%%%%%%%%%%%%%%%%%%%%%%%%%%%%%%%%%%
\section{Figuras e Tabelas}

%%%%%%%%%%%%%%%%%%%%%%%%%%%%%%%%%%%%%%%%%%%%%%%%%%%%%%%%%%%%%%%%%%%%%%%%%%%%%%%
%%%%%%%%%%%%%%%%%%%%%%%%%%%%%%%%%%%%%%%%%%%%%%%%%%%%%%%%%%%%%%%%%%%%%%%%%%%%%%%
%%%%%%%%%%%%%%%%%%%%%%%%%%%%%%%%%%%%%%%%%%%%%%%%%%%%%%%%%%%%%%%%%%%%%%%%%%%%%%%
\begin{frame}{Resumo}
\begin{multicols}{2}
\tableofcontents[currentsection]
\end{multicols}
\end{frame}

%%%%%%%%%%%%%%%%%%%%%%%%%%%%%%%%%%%%%%%%%%%%%%%%%%%%%%%%%%%%%%%%%%%%%%%%%%%%%%%
%%%%%%%%%%%%%%%%%%%%%%%%%%%%%%%%%%%%%%%%%%%%%%%%%%%%%%%%%%%%%%%%%%%%%%%%%%%%%%%
%%%%%%%%%%%%%%%%%%%%%%%%%%%%%%%%%%%%%%%%%%%%%%%%%%%%%%%%%%%%%%%%%%%%%%%%%%%%%%%
\subsection{Figuras}
\begin{frame}[fragile]{\insertsubsection}
\begin{itemize}
\item O pacote \bftt{graphicx} é necessário, já que provê o comando
\cmdbs{includegraphics}.
\item Formatos de figuras que tem suporte incluem JPEG, PNG e PDF (em geral). 
\end{itemize}
\begin{exampletwouptiny}
\includegraphics[
  width=0.5\textwidth]{gerbil}

\includegraphics[
  width=0.3\textwidth,
  angle=270]{gerbil}
\end{exampletwouptiny}

\tiny{Licença para imagem: \href{https://pixabay.com/en/animal-apple-attractive-beautiful-1239390/}{CC0}}
\end{frame}

%%%%%%%%%%%%%%%%%%%%%%%%%%%%%%%%%%%%%%%%%%%%%%%%%%%%%%%%%%%%%%%%%%%%%%%%%%%%%%%
%%%%%%%%%%%%%%%%%%%%%%%%%%%%%%%%%%%%%%%%%%%%%%%%%%%%%%%%%%%%%%%%%%%%%%%%%%%%%%%
%%%%%%%%%%%%%%%%%%%%%%%%%%%%%%%%%%%%%%%%%%%%%%%%%%%%%%%%%%%%%%%%%%%%%%%%%%%%%%%
\begin{frame}[fragile]{Interlúdio: Argumentos opcionais}
\begin{itemize}
\item Usamos colchetes  \keystrokebftt{[} \keystrokebftt{]} para inserir 
argumentos opcionais, ao invés de chaves\keystrokebftt{\{} \keystrokebftt{\}}.
\item \cmdbs{includgraphics} aceita argumentos opcionais que permitem você transformar
 a imagem que está sendo incluída. Por exemplo,  \bftt{width=0.3\cmdbs{textwidth}} faz 
 com que a imagem tenha 30\% da largura do texto  (\cmdbs{textwidth}).
\item \cmdbs{documentclass} também aceita argumenos opcionais. Exemplo:
\mint{latex}|\documentclass[12pt,twocolumn]{article}|
% \vskip 3ex
faz com que o texto torne-se maior (12pt) e mostra o texto em duas colunas.
\item Onde você encontra informações sobre isso? Veja os \emph{slides} no final 
dessa apresentação com \emph{links} para mais informações.
\end{itemize}
\end{frame}

%%%%%%%%%%%%%%%%%%%%%%%%%%%%%%%%%%%%%%%%%%%%%%%%%%%%%%%%%%%%%%%%%%%%%%%%%%%%%%%
%%%%%%%%%%%%%%%%%%%%%%%%%%%%%%%%%%%%%%%%%%%%%%%%%%%%%%%%%%%%%%%%%%%%%%%%%%%%%%%
%%%%%%%%%%%%%%%%%%%%%%%%%%%%%%%%%%%%%%%%%%%%%%%%%%%%%%%%%%%%%%%%%%%%%%%%%%%%%%%
\subsection[fragile]{Objetos flutuantes}
\begin{frame}{\insertsubsection}
\begin{itemize}
\item Permita que o \LaTeX{} decida onde a figura deve aparecer no texto
(ela pode  ``flutuar'').
\item Você pode colocar  legenda para uma figura usando \cmdbs{caption}, a qual pode ser referenciada  usando \cmdbs{ref}.
\end{itemize}
\begin{minipage}{0.55\linewidth}
\inputminted[fontsize=\scriptsize,frame=single,resetmargins]{latex}%
  {media-graphics.tex}
\end{minipage}
\begin{minipage}{0.35\linewidth}
\includegraphics[width=\textwidth,clip,trim=2in 5in 3in 1in]{media-graphics.pdf}
\end{minipage}
\end{frame}

%%%%%%%%%%%%%%%%%%%%%%%%%%%%%%%%%%%%%%%%%%%%%%%%%%%%%%%%%%%%%%%%%%%%%%%%%%%%%%%
%%%%%%%%%%%%%%%%%%%%%%%%%%%%%%%%%%%%%%%%%%%%%%%%%%%%%%%%%%%%%%%%%%%%%%%%%%%%%%%
%%%%%%%%%%%%%%%%%%%%%%%%%%%%%%%%%%%%%%%%%%%%%%%%%%%%%%%%%%%%%%%%%%%%%%%%%%%%%%%
\subsection{Tabelas}
\begin{frame}[fragile]{\insertsubsection}
\begin{itemize}
\item Tabelas em \LaTeX{} precisam de atenção.
\item Use o ambiente \bftt{tabular} do pacote \bftt{tabularx}.
\item Os argumento que especificam o alinhamento das colunas: \textbf{l} para esquerda (\emph{left}), \textbf{c} para centralizar e \textbf{r} para direita (\emph{right}).
\begin{exampletwouptiny}
\begin{tabular}{lcr}
Item    & Qtd. & Unid. \$ \\
Celular & 1    & 399,99  \\
Capa    & 2    & 99,99  \\
Cabo    & 3    & 19,99   \\
\end{tabular}
\end{exampletwouptiny}
\item É necessário especificar as linhas verticais. Para as linhas horizontais,  use o
comando \cmdbs{hline}.
\begin{exampletwouptiny}
\begin{tabular}{|l|c|r|} \hline
Item    & Qtd. & Unid. \$ \\\hline
Celular & 1    & 399,99  \\
Capa    & 2    & 99,99  \\
Cabo    & 3    & 19,99   \\\hline
\end{tabular}
\end{exampletwouptiny}
\item Use o ``E comercial''  \keystrokebftt{\&} para separar colunas e barras
invertidas duplas \keystrokebftt{\bs}\keystrokebftt{\bs} para iniciar uma nova
linha (como no ambiente  \bftt{align*} que vimos na parte 1).
\end{itemize}
\end{frame}

%%%%%%%%%%%%%%%%%%%%%%%%%%%%%%%%%%%%%%%%%%%%%%%%%%%%%%%%%%%%%%%%%%%%%%%%%%%%%%%
%%%%%%%%%%%%%%%%%%%%%%%%%%%%%%%%%%%%%%%%%%%%%%%%%%%%%%%%%%%%%%%%%%%%%%%%%%%%%%%
%%%%%%%%%%%%%%%%%%%%%%%%%%%%%%%%%%%%%%%%%%%%%%%%%%%%%%%%%%%%%%%%%%%%%%%%%%%%%%%
\addtocontents{toc}{\newpage}
\section{Referências bibliográficas}

%%%%%%%%%%%%%%%%%%%%%%%%%%%%%%%%%%%%%%%%%%%%%%%%%%%%%%%%%%%%%%%%%%%%%%%%%%%%%%%
%%%%%%%%%%%%%%%%%%%%%%%%%%%%%%%%%%%%%%%%%%%%%%%%%%%%%%%%%%%%%%%%%%%%%%%%%%%%%%%
%%%%%%%%%%%%%%%%%%%%%%%%%%%%%%%%%%%%%%%%%%%%%%%%%%%%%%%%%%%%%%%%%%%%%%%%%%%%%%%
\begin{frame}{Resumo}
\begin{multicols}{2}
\tableofcontents[currentsection]
\end{multicols}
\end{frame}
 
%%%%%%%%%%%%%%%%%%%%%%%%%%%%%%%%%%%%%%%%%%%%%%%%%%%%%%%%%%%%%%%%%%%%%%%%%%%%%%%
%%%%%%%%%%%%%%%%%%%%%%%%%%%%%%%%%%%%%%%%%%%%%%%%%%%%%%%%%%%%%%%%%%%%%%%%%%%%%%%
%%%%%%%%%%%%%%%%%%%%%%%%%%%%%%%%%%%%%%%%%%%%%%%%%%%%%%%%%%%%%%%%%%%%%%%%%%%%%%%
\subsection{bib\TeX}
\begin{frame}[fragile]{\insertsubsection{} 1} 
\begin{itemize}
\item Coloque suas referências em um arquivo  \bftt{.bib}:
\inputminted[fontsize=\tiny,frame=single]{latex}{bib-example.bib}
\item A maioria dos \emph{softwares} de administração de referências exporta para o formato \bftt{bibtex}.
\end{itemize}
\end{frame}

%%%%%%%%%%%%%%%%%%%%%%%%%%%%%%%%%%%%%%%%%%%%%%%%%%%%%%%%%%%%%%%%%%%%%%%%%%%%%%%
%%%%%%%%%%%%%%%%%%%%%%%%%%%%%%%%%%%%%%%%%%%%%%%%%%%%%%%%%%%%%%%%%%%%%%%%%%%%%%%
%%%%%%%%%%%%%%%%%%%%%%%%%%%%%%%%%%%%%%%%%%%%%%%%%%%%%%%%%%%%%%%%%%%%%%%%%%%%%%%
\begin{frame}[fragile]{\insertsubsection{} 2}
\begin{itemize}
\item Cada entrada do arquivo the \bftt{.bib} tem uma \emph{chave} que você pode
usar para referenciá-lo no documento. Por exemplo.  \bftt{Valente1999} é
a chave para citar o artigo:
\begin{minted}[fontsize=\small,frame=single]{latex}
@Article{Valente1999,
  author = {Wagner Rodrigues Valente},
  ...
}
\end{minted} 
\item É uma boa ideia usar chaves baseadas no autor, ano ou título do trabalho.
\item \LaTeX{} pode formatar automaticamente suas citações e gerar uma lista de
referências com os mais conhecidos padrões de estilos ou com o seu próprio.
\item Para o português brasileiro, há um conjunto de pacotes chamado 
\href{http://www.abntex.net.br/}{\bftt{abntex2}} que
formata as referências --- e o texto também --- segundo o padrão ABNT. 
\end{itemize}
\end{frame}

%%%%%%%%%%%%%%%%%%%%%%%%%%%%%%%%%%%%%%%%%%%%%%%%%%%%%%%%%%%%%%%%%%%%%%%%%%%%%%%
%%%%%%%%%%%%%%%%%%%%%%%%%%%%%%%%%%%%%%%%%%%%%%%%%%%%%%%%%%%%%%%%%%%%%%%%%%%%%%%
%%%%%%%%%%%%%%%%%%%%%%%%%%%%%%%%%%%%%%%%%%%%%%%%%%%%%%%%%%%%%%%%%%%%%%%%%%%%%%%
\begin{frame}[fragile]{\insertsubsection{} turbinado --- \bftt{biblatex}}
\begin{itemize}
\item Use o pacote \bftt{biblatex}  com opções \bftt{style} para formatar as referências (\bftt{numeric} é a preferida dos matemáticos) e  \cmdbs{cite} e \cmdbs{textcite} para citar no texto.
\item Insira o comando \cmdbs{addbibresource} para incluir seu arquivo de referências e \cmdbs{printbibliography} para imprimi-las.
\end{itemize}
\begin{minipage}{0.5\linewidth}
\inputminted[fontsize=\tiny,frame=single,resetmargins]{latex}%
  {bib-example.tex}
\end{minipage}
\begin{minipage}{0.4\linewidth}
\includegraphics[width=\textwidth,clip,trim=1.8in 5in 1.5in 1in]{bib-example.pdf}
\end{minipage} 
\end{frame}

%%%%%%%%%%%%%%%%%%%%%%%%%%%%%%%%%%%%%%%%%%%%%%%%%%%%%%%%%%%%%%%%%%%%%%%%%%%%%%%
%%%%%%%%%%%%%%%%%%%%%%%%%%%%%%%%%%%%%%%%%%%%%%%%%%%%%%%%%%%%%%%%%%%%%%%%%%%%%%%
%%%%%%%%%%%%%%%%%%%%%%%%%%%%%%%%%%%%%%%%%%%%%%%%%%%%%%%%%%%%%%%%%%%%%%%%%%%%%%%
\begin{frame}[fragile]{\insertsubsection{} turbinado II --- \bftt{biblatex} e  norma ABNT}
\begin{itemize}
\item  A partir do TeXLive 2016 está disponível o estilo \bftt{abnt} no \bftt{biblatex}.
\item Basta colocar  como opção \bftt{style = abnt}. Todo o resto, o \LaTeX{} faz pra você.
\end{itemize}
\begin{minipage}{0.5\linewidth}
\inputminted[fontsize=\tiny,frame=single,resetmargins]{latex}%
  {bib-example-abnt.tex}
\end{minipage}
\begin{minipage}{0.4\linewidth}
\includegraphics[width=\textwidth,clip,trim=1.8in 5in 1.5in 1in]{bib-example-abnt.pdf}
\end{minipage} 
\end{frame}

%%%%%%%%%%%%%%%%%%%%%%%%%%%%%%%%%%%%%%%%%%%%%%%%%%%%%%%%%%%%%%%%%%%%%%%%%%%%%%%
%%%%%%%%%%%%%%%%%%%%%%%%%%%%%%%%%%%%%%%%%%%%%%%%%%%%%%%%%%%%%%%%%%%%%%%%%%%%%%%
%%%%%%%%%%%%%%%%%%%%%%%%%%%%%%%%%%%%%%%%%%%%%%%%%%%%%%%%%%%%%%%%%%%%%%%%%%%%%%%
\subsection{Exercício}
\begin{frame}[fragile]{Exercício: Juntando tudo}

Adicione uma imagem, uma tabela  e várias bibliografias\footnote{Inclua tipos diferentes como artigo, livro, teses, etc.} ao artigo do exercício anterior.

\begin{enumerate}
\item Baixe os arquivos abaixo no seu computador.

\begin{center}
\fbox{\href{\fileuri/gerbil.jpg?dl=1}{Clique para baixar a imagem}}

\fbox{\href{\fileuri/bib-exercise.bib?dl=1}{Clique para baixar o arquivo .bib}}
\end{center}

\item Adicione-os ao  Overleaf (use o menu de arquivos).


\end{enumerate}
\end{frame}

%%%%%%%%%%%%%%%%%%%%%%%%%%%%%%%%%%%%%%%%%%%%%%%%%%%%%%%%%%%%%%%%%%%%%%%%%%%%%%%
%%%%%%%%%%%%%%%%%%%%%%%%%%%%%%%%%%%%%%%%%%%%%%%%%%%%%%%%%%%%%%%%%%%%%%%%%%%%%%%
%%%%%%%%%%%%%%%%%%%%%%%%%%%%%%%%%%%%%%%%%%%%%%%%%%%%%%%%%%%%%%%%%%%%%%%%%%%%%%%
\section{E agora?}

%%%%%%%%%%%%%%%%%%%%%%%%%%%%%%%%%%%%%%%%%%%%%%%%%%%%%%%%%%%%%%%%%%%%%%%%%%%%%%%
%%%%%%%%%%%%%%%%%%%%%%%%%%%%%%%%%%%%%%%%%%%%%%%%%%%%%%%%%%%%%%%%%%%%%%%%%%%%%%%
%%%%%%%%%%%%%%%%%%%%%%%%%%%%%%%%%%%%%%%%%%%%%%%%%%%%%%%%%%%%%%%%%%%%%%%%%%%%%%%
\begin{frame}{Resumo}
\begin{multicols}{2}
\tableofcontents[currentsection]
\end{multicols}
\end{frame}

%%%%%%%%%%%%%%%%%%%%%%%%%%%%%%%%%%%%%%%%%%%%%%%%%%%%%%%%%%%%%%%%%%%%%%%%%%%%%%%
%%%%%%%%%%%%%%%%%%%%%%%%%%%%%%%%%%%%%%%%%%%%%%%%%%%%%%%%%%%%%%%%%%%%%%%%%%%%%%%
%%%%%%%%%%%%%%%%%%%%%%%%%%%%%%%%%%%%%%%%%%%%%%%%%%%%%%%%%%%%%%%%%%%%%%%%%%%%%%%
\subsection{Mais coisas legais}
\begin{frame}[fragile]{\insertsubsection}
\begin{itemize}
\item Adicione o comando \cmdbs{tableofcontents} para gerar o sumário automaticamente a partir dos comandos  \cmdbs{section}.

\item Mude a classe de documentos em \cmdbs{documentclass} para
\mint{latex}!\documentclass{scrartcl} %Koma-Script!
ou
\mint{latex}!\documentclass[12pt]{IEEEtran} %IEEE Transactions!

\item Defina o seu próprio comando de uma equação complicada:
\begin{exampletwouptiny}
\newcommand{\rperf}{%
  \rho_{\text{perf}}}
$$
\rperf = {\bf c}'{\bf X} + \varepsilon
$$
\end{exampletwouptiny}
\end{itemize}
\end{frame}

%%%%%%%%%%%%%%%%%%%%%%%%%%%%%%%%%%%%%%%%%%%%%%%%%%%%%%%%%%%%%%%%%%%%%%%%%%%%%%%
%%%%%%%%%%%%%%%%%%%%%%%%%%%%%%%%%%%%%%%%%%%%%%%%%%%%%%%%%%%%%%%%%%%%%%%%%%%%%%%
%%%%%%%%%%%%%%%%%%%%%%%%%%%%%%%%%%%%%%%%%%%%%%%%%%%%%%%%%%%%%%%%%%%%%%%%%%%%%%%
\subsection{Mais pacotes legais}
\begin{frame}{\insertsubsection}
\begin{itemize}
\item \bftt{beamer}: para apresentações (como esta aqui!)
\item \bftt{todonotes}: comentários e gerenciamento listas TODO
\item \bftt{tikz}: faça figuras sensacionais
\item \bftt{pgfplots}: crie  gráficos em \LaTeX
\item \bftt{listings}: imprima seu código fonte em \LaTeX
\item \bftt{spreadtab}: crie planilhas em  \LaTeX
\item \bftt{gchords}, \bftt{guitar}: acordes para violão e tablaturas
\item \bftt{cwpuzzle}: para palavras cruzadas
\end{itemize}
Veja \url{https://www.overleaf.com/latex/examples} ou \url{http://texample.net} para
exemplos da maioria desses pacotes\footnote{Em inglês.}.
\end{frame}

%%%%%%%%%%%%%%%%%%%%%%%%%%%%%%%%%%%%%%%%%%%%%%%%%%%%%%%%%%%%%%%%%%%%%%%%%%%%%%%
%%%%%%%%%%%%%%%%%%%%%%%%%%%%%%%%%%%%%%%%%%%%%%%%%%%%%%%%%%%%%%%%%%%%%%%%%%%%%%%
%%%%%%%%%%%%%%%%%%%%%%%%%%%%%%%%%%%%%%%%%%%%%%%%%%%%%%%%%%%%%%%%%%%%%%%%%%%%%%%
\subsection{Instalando \LaTeX{}}
\begin{frame}{\insertsubsection}
\begin{itemize}
\item Para rodar o  \LaTeX{} no seu próprio computador, você vai precisar de uma \emph{distribuição} do \LaTeX{}. Uma distribuição inclui  um programa \bftt{latex} e (tipicamente) alguns milhares de pacotes.
\begin{itemize}
\item No Windows: \href{http://miktex.org/}{Mik\TeX} ou \href{http://tug.org/texlive/}{\TeX Live}
\item No Linux: \href{http://tug.org/texlive/}{\TeX Live}
\item No Mac: \href{http://tug.org/mactex/}{Mac\TeX}
\end{itemize} 
\item Você também vai querer utilizar um editor para \LaTeX{}. Veja \url{en.wikipedia.org/wiki/Comparison_of_TeX_editors} para uma lista em inglês de opções.
\item Você também vai precisar saber mais sobre como \bftt{latex} e suas ferramentas relacionadas funcionam --- veja fontes de recursos no próximo \emph{slide}.
\end{itemize}
\end{frame}

%%%%%%%%%%%%%%%%%%%%%%%%%%%%%%%%%%%%%%%%%%%%%%%%%%%%%%%%%%%%%%%%%%%%%%%%%%%%%%%
%%%%%%%%%%%%%%%%%%%%%%%%%%%%%%%%%%%%%%%%%%%%%%%%%%%%%%%%%%%%%%%%%%%%%%%%%%%%%%%
%%%%%%%%%%%%%%%%%%%%%%%%%%%%%%%%%%%%%%%%%%%%%%%%%%%%%%%%%%%%%%%%%%%%%%%%%%%%%%%
\subsection{Recursos}
\begin{frame}{\insertsubsection}
\begin{itemize}
\item \href{http://en.wikibooks.org/wiki/LaTeX}{The \LaTeX{} Wikibook} ---
tutoriais excelentes e material de referência.
\item \href{http://tex.stackexchange.com/}{\TeX{} Stack Exchange} --- pergunte questões e tenha respostas excelentes e rápidas.
\item \href{http://www.latex-community.org/}{\LaTeX{} Community} --- um imenso fórum \emph{online}
\item \href{http://ctan.org/}{Comprehensive \TeX{} Archive Network (CTAN)} ---
mais de quatro mil pacotes e sua documentação


\item \href{http://latexbr.blogspot.com.br/}{\LaTeX{} BR --- blog com dicas em português}
\item Google em geral vai levá-lo para um dos \emph{links} acima.
\end{itemize}
\end{frame}

%%%%%%%%%%%%%%%%%%%%%%%%%%%%%%%%%%%%%%%%%%%%%%%%%%%%%%%%%%%%%%%%%%%%%%%%%%%%%%%
%%%%%%%%%%%%%%%%%%%%%%%%%%%%%%%%%%%%%%%%%%%%%%%%%%%%%%%%%%%%%%%%%%%%%%%%%%%%%%%
%%%%%%%%%%%%%%%%%%%%%%%%%%%%%%%%%%%%%%%%%%%%%%%%%%%%%%%%%%%%%%%%%%%%%%%%%%%%%%%
\begin{frame}
\begin{center}
Obrigado, e capriche nos seus próximos  \TeX{}tos!
\end{center}
\end{frame}

\end{document}

% -- latex understands words, sentences and paragraphs

Words are separated by one or more spaces.  Paragraphs are separated by
one or more blank lines.  The output is not affected by adding extra
spaces or extra blank lines to the input file.

Double quotes are typed like this: ``quoted text''.
Single quotes are typed like this: `single-quoted text'.

Emphasized text is typed like this: \emph{this is emphasized}.
Bold       text is typed like this: \textbf{this is bold}.

-- Adding structure to your document

\section{Hello}

\subsection{World}

\subsection{Foo}

\subsubsection*{Stuff} % star form

\subsubsection*{Results}

-- Labels and cross-references

\label{sec:intro}
\label{sec:method}
\ref{sec:method}

--> maybe introduce the prettyref package here.

-- Mathematics

Inline mathematics: $x + y < 7$.

'Displayed' mathematics:
\begin{equation}
\end{equation}

\begin{equation*}
\end{equation*}

\begin{align}
\end{align}

-- Figures

- Need the graphicx package.

- here we can start introducing options

\includegraphics[width=\textwidth]{}

- where do you find out about these options? --> link to the Wikibook

-- Floating Figures

\begin{figure}
\includegraphics{...}
\caption{\label{}Here is a caption.}
\end{figure}

-- Tables

- not the nicest part of LaTeX

\usepackage{tabularx}

\begin{tabular}{llr}
Item & Quantity & Price (\$) & Amount
Widget & 1 & 
\end{tabular}

Bonus points: check out the fp package and the spreadtab package.

-- Document Classes

a .cls file

article

some journal templates come with one

-- Bibliographies



-- For Typesetting Geeks

- dashes: -, --, ---

- ellipsis.

- controlling spaces: ~, \ , \,, \@

- spacing after periods (et al., etc.)

- Nested quotation marks: ``\,`
\vskip 2ex
\item Use the \emph{star form} to display an equation without a number.
\begin{exampletwouptiny}
\begin{equation*}
F(x) = \int_{a}^{x}{f(t) dt}
\end{equation*}
\end{exampletwouptiny}

\begin{itemize}
\item \bftt{equation} and \bftt{equation*} are called \emph{environments}.
\begin{itemize}
  \item The \cmdbs{begin} and \cmdbs{end} commands define the environment.
  \item The \cmd{\$} also starts and ends an environment.
  \item Some commands are defined only within certain environments.
  \item Some commands behave differently in different environments.
\end{itemize}
\end{itemize}
\end{block}
\begin{center}
\fbox{\href{http://ctan.org/}{The Comprehensive \TeX Archive Network (CTAN)}}
\end{center}

%%%%%%%%%%%%%%%%%%%%%%%%%%%%%%%%%%%%%%%%%%%%%%%%%%%%%%%%%%%%%%%%%%%%%%%%%%%%%%%
%%%%%%%%%%%%%%%%%%%%%%%%%%%%%%%%%%%%%%%%%%%%%%%%%%%%%%%%%%%%%%%%%%%%%%%%%%%%%%%
%%%%%%%%%%%%%%%%%%%%%%%%%%%%%%%%%%%%%%%%%%%%%%%%%%%%%%%%%%%%%%%%%%%%%%%%%%%%%%%
\subsection{Typography tweaks}
\begin{frame}{\insertsubsection}
\begin{tabular}{lll}
& character name & used mainly for \ldots \\\hline
\bftt{\bs} & backslash                 & commands, tables \\
\bftt{\{}  & open brace                & commands \\
\bftt{\}}  & close brace               & commands \\
\bftt{\%}  & percent sign              & comments \\
\bftt{\#}  & hash (pound / sharp) sign & custom commands \\
\bftt{\$}  & dollar sign               & equations \\
\bftt{\_}  & underscore                & equations (subscripts) \\
\bftt{\^}  & caret                     & equations (superscripts) \\
\bftt{\&}  & ampersand                 & tables \\
\bftt{\~}  & tilde                     & spacing \\
\end{tabular}
\end{frame}

%\item We've used several environments:
%\vskip 1ex
%{\scriptsize
%\begin{tabular}{ll}
%\cmdbs{begin}\bftt{\{document\}}\ldots\cmdbs{end}\bftt{\{document\}} &
%  document environment \\
%\cmdbs{begin}\bftt{\{itemize\}}\ldots\cmdbs{end}\bftt{\{itemize\}} &
%  itemized list environment \\
%\bftt{\$\ldots\$}     & \emph{in-text} math environment \\
%\bftt{\$\$\ldots\$\$} & \emph{displayed} math environment \\
%\cmdbs{begin}\bftt{\{equation\}}\ldots\cmdbs{end}\bftt{\{equation\}} &
%  displayed math environment w/ number
%\end{tabular}
%}

