\documentclass[12pt]{article}
\usepackage[brazil]{babel}
 \usepackage[T1]{fontenc} % Adicionado para poder usar os Caracteres Portugueses
\usepackage[utf8]{inputenc} % Adicionado para poder usar os Caracteres Portugueses

\usepackage{url}

\begin{document}
Dez Segredos Para uma Boa Apresentação

Autor: Você

Introdução

O Texto constante neste exercício é uma versão traduzida e resumida do excelente artigo da autoria de Mark Schoeberl e Brian Toon:
\url{http://www.cgd.ucar.edu/cms/agu/scientific_talk.html}

Os Segredos

Compilei esta lista pessoal de "Segredos" depois de ouvir apresentações eficazes e ineficazes de diversos oradores. Esta lista não é completa - Com certeza algumas coisas ficaram de fora. Mas, esta lista contêm 90% do que se deve saber e  fazer.

1) Prepare o seu material cuidadosamente e logicamente. Conte uma história.

2) Pratique o sua fala. Falta de preparação não é uma desculpa.

3) Não tenha  material demais. Bons Oradores apresentam um ou dois pontos principais e focam-se neles.

4) Evite equações. Diz-se que para cada equação em sua apresentação, o número de pessoas que lhe compreendem passa para metade. Isto é, sendo q o número de equações na apresentação, o número de pessoas que prestam atenção na apresentação é dado por:

n = gamma (1/2) elevado a q

onde gama é a constante de proporcionalidade [1,4].

5) A conclusão deve conter apenas os pontos chave. As pessoas não se lembram de mais que 2 ou 3 coisas de uma apresentação, principalmente se tiverem ouvido várias apresentações em grandes conferências.

6) Fale para audiência e não para a tela. Um dos erros mais comuns é o orador falar virado de costas para audiência. 

7) Evite fazer sons que distraiam audiência. Evite os "Ummm" ou "Ahhh" entre frases.

8) Melhore os seus gráficos. Uma pequena lista de dicas para melhorar gráficos numa apresentação:

* Use  fontes grande.

* Mantenha os gráficos simples. Não mostre gráficos sem necessidade.

* Use cores.

9) Seja pessoal ao responder a questões [2,3].

10) Use humor sempre que possível. É fascinante como uma piada simples consegue fazer as pessoas rirem numa conferência científica.



Referências 


[1] W. Boyce e R. C. DiPrima. Equações Diferenciais elementares e Problemas de Valor de Contorno. 10a ed. Rio de Janeiro: LTC, 2010, p. 624.

[2] Gilbert Strang. Introdução à Álgebra Linear. 4a ed. Rio de Janeiro: LTC, 2013, pp. 166-167.

[3] D. Saviani. Educação. João Pessoa: Cortez/Autores Associados, 1980.



[4] Wagner Rodrigues Valente. Há 150 Anos Uma Querela sobre a Geometria Elementar no Brasil: algumas Cenas dos Bastidores da Produção do Saber Escolar. Em: BOLEMA 12 (13 1999), pp. 44–61.


\end{document}